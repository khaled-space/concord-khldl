%%
%% Beginning of file 'ms.tex'
%%
%% Copied from sample61.tex by dkg 15.3.2018
%%
%% This is the paper describing the use and design of the concord suite of software
%% for burst observation-model comparisons
%%
%% See https://bitbucket.org/minbar/concord
%%
%% AASTeX is now based on Alexey Vikhlinin's emulateapj.cls 
%% (Copyright 2000-2015).  See the classfile for details.

%% AASTeX requires revtex4-1.cls (http://publish.aps.org/revtex4/) and
%% other external packages (latexsym, graphicx, amssymb, longtable, and epsf).
%% All of these external packages should already be present in the modern TeX 
%% distributions.  If not they can also be obtained at www.ctan.org.

%% The first piece of markup in an AASTeX v6.x document is the \documentclass
%% command. LaTeX will ignore any data that comes before this command. The 
%% documentclass can take an optional argument to modify the output style.
%% The command below calls the preprint style  which will produce a tightly 
%% typeset, one-column, single-spaced document.  It is the default and thus
%% does not need to be explicitly stated.
%%
%%
%% using aastex version 6.1
\documentclass{aastex61}

%% The default is a single spaced, 10 point font, single spaced article.
%% There are 5 other style options available via an optional argument. They
%% can be envoked like this:
%%
%% \documentclass[argument]{aastex61}
%% 
%% where the arguement options are:
%%
%%  twocolumn   : two text columns, 10 point font, single spaced article.
%%                This is the most compact and represent the final published
%%                derived PDF copy of the accepted manuscript from the publisher
%%  manuscript  : one text column, 12 point font, double spaced article.
%%  preprint    : one text column, 12 point font, single spaced article.  
%%  preprint2   : two text columns, 12 point font, single spaced article.
%%  modern      : a stylish, single text column, 12 point font, article with
%% 		  wider left and right margins. This uses the Daniel
%% 		  Foreman-Mackey and David Hogg design.
%%
%% Note that you can submit to the AAS Journals in any of these 6 styles.
%%
%% There are other optional arguments one can envoke to allow other stylistic
%% actions. The available options are:
%%
%%  astrosymb    : Loads Astrosymb font and define \astrocommands. 
%%  tighten      : Makes baselineskip slightly smaller, only works with 
%%                 the twocolumn substyle.
%%  times        : uses times font instead of the default
%%  linenumbers  : turn on lineno package.
%%  trackchanges : required to see the revision mark up and print its output
%%  longauthor   : Do not use the more compressed footnote style (default) for 
%%                 the author/collaboration/affiliations. Instead print all
%%                 affiliation information after each name. Creates a much
%%                 long author list but may be desirable for short author papers
%%
%% these can be used in any combination, e.g.
%%
%% \documentclass[twocolumn,linenumbers,trackchanges]{aastex61}

%% AASTeX v6.* now includes \hyperref support. While we have built in specific
%% defaults into the classfile you can manually override them with the
%% \hypersetup command. For example,
%%
%%\hypersetup{linkcolor=red,citecolor=green,filecolor=cyan,urlcolor=magenta}
%%
%% will change the color of the internal links to red, the links to the
%% bibliography to green, the file links to cyan, and the external links to
%% magenta. Additional information on \hyperref options can be found here:
%% https://www.tug.org/applications/hyperref/manual.html#x1-40003

%% If you want to create your own macros, you can do so
%% using \newcommand. Your macros should appear before
%% the \begin{document} command.
%%
\newcommand{\vdag}{(v)^\dagger}
\newcommand\aastex{AAS\TeX}
\newcommand\latex{La\TeX}

%% Reintroduced the \received and \accepted commands from AASTeX v5.2

%\received{July 1, 2016}
%\revised{September 27, 2016}
%\accepted{\today}

%% Command to document which AAS Journal the manuscript was submitted to.
%% Adds "Submitted to " the arguement.
\submitjournal{ApJ}

%% Mark up commands to limit the number of authors on the front page.
%% Note that in AASTeX v6.1 a \collaboration call (see below) counts as
%% an author in this case.
%
%\AuthorCollaborationLimit=3
%
%% Will only show Schwarz, Muench and "the AAS Journals Data Scientist 
%% collaboration" on the front page of this example manuscript.
%%
%% Note that all of the author will be shown in the published article.
%% This feature is meant to be used prior to acceptance to make the
%% front end of a long author article more manageable. Please do not use
%% this functionality for manuscripts with less than 20 authors. Conversely,
%% please do use this when the number of authors exceeds 40.
%%
%% Use \allauthors at the manuscript end to show the full author list.
%% This command should only be used with \AuthorCollaborationLimit is used.

%% The following command can be used to set the latex table counters.  It
%% is needed in this document because it uses a mix of latex tabular and
%% AASTeX deluxetables.  In general it should not be needed.
%\setcounter{table}{1}

%%%%%%%%%%%%%%%%%%%%%%%%%%%%%%%%%%%%%%%%%%%%%%%%%%%%%%%%%%%%%%%%%%%%%%%%%%%%%%%%
%%
%% The following section outlines numerous optional output that
%% can be displayed in the front matter or as running meta-data.
%%
%% If you wish, you may supply running head information, although
%% this information may be modified by the editorial offices.
\shorttitle{Burst model-observation comparisons}
\shortauthors{Galloway et al.}
%%
%% You can add a light gray and diagonal water-mark to the first page 
%% with this command:
% \watermark{text}
%% where "text", e.g. DRAFT, is the text to appear.  If the text is 
%% long you can control the water-mark size with:
%  \setwatermarkfontsize{dimension}
%% where dimension is any recognized LaTeX dimension, e.g. pt, in, etc.
%%
%%%%%%%%%%%%%%%%%%%%%%%%%%%%%%%%%%%%%%%%%%%%%%%%%%%%%%%%%%%%%%%%%%%%%%%%%%%%%%%%

%% This is the end of the preamble.  Indicate the beginning of the
%% manuscript itself with \begin{document}.

\begin{document}

\title{Thermonuclear burst model-observation comparisons with {\sc concord}}

%% LaTeX will automatically break titles if they run longer than
%% one line. However, you may use \\ to force a line break if
%% you desire. In v6.1 you can include a footnote in the title.

%% A significant change from earlier AASTEX versions is in the structure for 
%% calling author and affilations. The change was necessary to implement 
%% autoindexing of affilations which prior was a manual process that could 
%% easily be tedious in large author manuscripts.
%%
%% The \author command is the same as before except it now takes an optional
%% arguement which is the 16 digit ORCID. The syntax is:
%% \author[xxxx-xxxx-xxxx-xxxx]{Author Name}
%%
%% This will hyperlink the author name to the author's ORCID page. Note that
%% during compilation, LaTeX will do some limited checking of the format of
%% the ID to make sure it is valid.
%%
%% Use \affiliation for affiliation information. The old \affil is now aliased
%% to \affiliation. AASTeX v6.1 will automatically index these in the header.
%% When a duplicate is found its index will be the same as its previous entry.
%%
%% Note that \altaffilmark and \altaffiltext have been removed and thus 
%% can not be used to document secondary affiliations. If they are used latex
%% will issue a specific error message and quit. Please use multiple 
%% \affiliation calls for to document more than one affiliation.
%%
%% The new \altaffiliation can be used to indicate some secondary information
%% such as fellowships. This command produces a non-numeric footnote that is
%% set away from the numeric \affiliation footnotes.  NOTE that if an
%% \altaffiliation command is used it must come BEFORE the \affiliation call,
%% right after the \author command, in order to place the footnotes in
%% the proper location.
%%
%% Use \email to set provide email addresses. Each \email will appear on its
%% own line so you can put multiple email address in one \email call. A new
%% \correspondingauthor command is available in V6.1 to identify the
%% corresponding author of the manuscript. It is the author's responsibility
%% to make sure this name is also in the author list.
%%
%% While authors can be grouped inside the same \author and \affiliation
%% commands it is better to have a single author for each. This allows for
%% one to exploit all the new benefits and should make book-keeping easier.
%%
%% If done correctly the peer review system will be able to
%% automatically put the author and affiliation information from the manuscript
%% and save the corresponding author the trouble of entering it by hand.

\correspondingauthor{Duncan K. Galloway}
\email{duncan.galloway@monash.edu}

\author[0000-0002-6558-5121]{Duncan K. Galloway}
\affil{School of Physics \& Astronomy \\
Monash University \\
Clayton, VIC 3800, Australia}

%\author{August Muench}
%\affiliation{American Astronomical Society \\
%2000 Florida Ave., NW, Suite 300 \\
%Washington, DC 20009-1231, USA}
%\collaboration{(AAS Journals Data Scientists collaboration)}
%
%\author{Butler Burton}
%\affiliation{National Radio Astronomy Observatory}
%\affiliation{AAS Journals Associate Editor-in-Chief}
%\nocollaboration
%
%\author{Amy Hendrickson}
%\altaffiliation{Creator of AASTeX v6.1}
%\affiliation{TeXnology Inc.}
%\collaboration{(LaTeX collaboration)}
%
%\author{Julie Steffen}
%\affiliation{AAS Director of Publishing}
%\affiliation{American Astronomical Society \\
%2000 Florida Ave., NW, Suite 300 \\
%Washington, DC 20009-1231, USA}
%
%\author{Jeff Lewandowski}
%\affiliation{IOP Senior Publisher for the AAS Journals}
%\affiliation{IOP Publishing, Washington, DC 20005}

%% Note that the \and command from previous versions of AASTeX is now
%% depreciated in this version as it is no longer necessary. AASTeX 
%% automatically takes care of all commas and "and"s between authors names.

%% AASTeX 6.1 has the new \collaboration and \nocollaboration commands to
%% provide the collaboration status of a group of authors. These commands 
%% can be used either before or after the list of corresponding authors. The
%% argument for \collaboration is the collaboration identifier. Authors are
%% encouraged to surround collaboration identifiers with ()s. The 
%% \nocollaboration command takes no argument and exists to indicate that
%% the nearby authors are not part of surrounding collaborations.

%% Mark off the abstract in the ``abstract'' environment. 
\begin{abstract}

% AAS Journals, the Astrophysical Journal (ApJ), the
%Astrophysical Journal Letters (ApJL), and Astronomical Journal (AJ), all
%have a 250 word limit for the abstract.  If you exceed this length the
%Editorial office will ask you to shorten it.

Thermonuclear bursts arise from unstable ignition of accumulated fuel on the surface of neutron stars in low-mass X-ray binaries. Comparison of observed bursts with numerical models is a key factor in determining the properties of the host neutron star.
%
Here we describe some commonly-used approaches for determining system parameters, including composition of the burst fuel, and introduce a new suite of software tools, {\sc concord}, intended to make comprehensive model-observation comparisons practical.

\end{abstract}

%% Keywords should appear after the \end{abstract} command. 
%% See the online documentation for the full list of available subject
%% keywords and the rules for their use.
\keywords{editorials, notices --- 
miscellaneous --- catalogs --- surveys}

%% From the front matter, we move on to the body of the paper.
%% Sections are demarcated by \section and \subsection, respectively.
%% Observe the use of the LaTeX \label
%% command after the \subsection to give a symbolic KEY to the
%% subsection for cross-referencing in a \ref command.
%% You can use LaTeX's \ref and \label commands to keep track of
%% cross-references to sections, equations, tables, and figures.
%% That way, if you change the order of any elements, LaTeX will
%% automatically renumber them.

%% We recommend that authors also use the natbib \citep
%% and \citet commands to identify citations.  The citations are
%% tied to the reference list via symbolic KEYs. The KEY corresponds
%% to the KEY in the \bibitem in the reference list below. 

\section{Introduction} \label{sec:intro}

%\latex\ \footnote{\url{http://www.latex-project.org/}} is a document markup
%language that is particularly well suited for the publication of
%mathematical and scientific articles \citep{lamport94}. \latex\ was written
%in 1985 by Leslie Lamport who based it on the \TeX\ typesetting language
%which itself was created by Donald E. Knuth in 1978.  In 1988 a suite of
%\latex\ macros were developed to investigate electronic submission and
%publication of AAS Journal articles \citep{1989BAAS...21..780H}.  Shortly
%afterwards, Chris Biemesdefer merged these macros and more into a \latex\
%2.08 style file called \aastex.  These early \aastex\ versions introduced
%many common commands and practices that authors take for granted today.
%Substantial revisions
%were made by Lee Brotzman and Pierre Landau when the package was updated to
%v4.0.  AASTeX v5.0, written in 1995 by Arthur Ogawa, upgraded to \latex\ 2e
%which uses the document class in lieu of a style file.  Other improvements
%to version 5 included hypertext support, landscape deluxetables and
%improved figure support to facilitate electronic submission.  
%\aastex\ v5.2 was released in 2005 and introduced additional graphics
%support plus new mark up to identifier astronomical objects, datasets and
%facilities.
%
%In 1996 Maxim Markevitch modified the AAS preprint style file, aaspp4.sty,
%to closely emulate the very tight, two column style of a typeset
%Astrophysical Journal article.  The result was emulateapj.sty.  A year
%later Alexey Vikhlinin took over development and maintenance.  In 2001 he
%converted emulateapj into a class file in \latex\ 2e and in 2003 Vikhlinin
%completely rewrote emulateapj based on the APS Journal's RevTEX class.
%
%During this time emulateapj gained growing acceptance in the astronomical
%community as it filled an author need to obtain an approximate number of
%manuscript pages prior to submission for cost and length estimates. The
%tighter typeset also had the added advantage of saving paper when printing
%out hard copies.
%
%Even though author publication charges are no longer based on print pages
%\footnote{see Section \ref{sec:pubcharge} in the Appendix for more details
%about how current article costs are calculated.} the emulateapj class file
%has proven to be extremely popular with AAS Journal authors.  An informal
%analysis of submitted \latex\ manuscripts in 2015 revealed that $\sim$65\%
%either called emulateapj or have a commented emulateapj classfile call
%indicating it was used at some stage of the manuscript construction.
%Clearly authors want to have access to a tightly typeset version of the
%article when corresponding with co-authors and for preprint submissions.
%
%When planning the next \aastex\ release the popularity of emulateapj played
%an important roll in the decision to drop the old base code and adopt and
%modify emulateapj for \aastex\ v6.+ instead.  The change brings \aastex\
%inline with what the majority of authors are already using while still
%delivering new and improved features.  \aastex\ v6.0 and v6.1 were written
%by Amy Hendrickson and released in January and October 2016, respectively.
%Some of the new features in v6.0 included:
%
%\begin{enumerate}
%\item improved citations for third party data repositories and software,
%\item easier construction of matrix figures consisting of multiple 
%encapsulated postscript (EPS) or portable document format (PDF) files,
%\item figure set mark up for large collections of similar figures,
%\item color mark up to easily enable/disable revised text highlighting,
%\item improved url support, and
%\item numerous table options such as the ability to hide columns, column 
%decimal alignment, automatic column math mode and numbering, plus splitting of
%wide tables.
%\end{enumerate}
%
%The new features in v6.1 are:
%
%\begin{enumerate}
%\item ORCID support for preprints,
%\item improved author, affiliation and collaboration mark up,
%\item reintroduced the old AASTeX v5.2 {\tt\string\received}, 
%      {\tt\string\revised}, {\tt\string\accepted}, and
%      {\tt\string\published} commands plus
%      added the new {\tt\string\submitjournal} command to document
%      which AAS Journal the manuscript was submitted to, plus
%\item new typeset style options.
%\end{enumerate}
%
%The rest of this article provides information and examples on how to create
%your own AAS Journal manuscript with v6.1.  Special emphasis is placed on
%how to use the full potential of \aastex\ v6+.  The next section describes
%the different manuscript styles available and how they differ from past
%releases.  Section \ref{sec:floats} describes how tables and figures are
%placed in a \latex\ document. Specific examples of tables, Section
%\ref{subsec:tables}, and figures, Section \ref{subsec:figures}, are also
%provided.  Section \ref{sec:displaymath} discusses how to display math and
%incorporate equations in a manuscript while Section \ref{sec:highlight}
%discuss how to use the new revision mark up.  The last section,
%\ref{sec:cite}, shows how recognize software and external data as first
%class references in the manuscript bibliography.  An appendix is included
%to show how to construct one and provide some information on how article
%charges are calculated.  Additional information is available both embedded
%in the comments of this \latex\ file and in the online documentation at
%\url{http://journals.aas.org/authors/aastex.html}.

\section{Observations and modelling} 
\label{sec:data}

\subsection{Burst observations}
\label{subsec:data}

The fundamental observables that we measure include the recurrence time $\Delta t$; the persistent flux $F_{\rm per}$, a measure of the accretion rate; and the burst bolometric flux lightcurve, $(t_i,F_{i})$. From the burst lightcurve we may integrate to estimate the burst fluence, $E_b$, which may be used to estimate the $\alpha$-value, the ratio of the burst to persistent flux: 
\begin{equation}
\alpha = \frac{\Delta t F_{\rm per} c_{\rm bol}}{E_B}
\end{equation}
where $c_{\rm bol}$ is the bolometric correction giving the inverse fraction of total persistent flux emitted in the instrumental band.

\subsection{Burst models}
\label{subsec:models}

Thermonuclear burst simulations \cite[e.g. with {\sc kepler};][]{woos04} are usually carried out in a plane-parallel grid with constant gravity, $g$. The principal input parameters include the accretion rate (per unit area), $\dot{m}$; the composition of the accreted fuel, usually quantified as $(X_0, Z_{\rm CNO})$ where $X_0$ is the mass fraction of hydrogen, and $Z_{\rm CNO}$ the mass fraction of CNO nuclei, which drive the hot-CNO cycle burning between bursts; and a parameter describing the degree of heating from below the model domain, usually labeled as ``base flux'' $Q_b$.

The model predictions include a recurrence time $\Delta t_{\rm pred}$, burst energy $E_{\rm pred}$ and lightcurve $(t_i,L_{{\rm pred},i})$ covering the extent of the burst.
%
The accretion rate $\dot{m}$ may also be converted to a persistent flux level for comparison with observations.


%The default style in \aastex\ v6.1 is a tight single column style, e.g.  10
%point font, single spaced.  The single column style is very useful for
%article with wide equations. It is also the easiest to style to work with
%since figures and tables, see Section \ref{sec:floats}, will span the
%entire page, reducing the need for address float sizing.
%
%To invoke a two column style similar to the what is produced in
%the published PDF copy use \\
%
%\noindent {\tt\string\documentclass[twocolumn]\{aastex61\}}. \\
%
%\noindent Note that in the two column style figures and tables will only
%span one column unless specifically ordered across both with the ``*'' flag,
%e.g. \\
%
%\noindent{\tt\string\begin\{figure*\}} ... {\tt\string\end\{figure*\}}, \\
%\noindent{\tt\string\begin\{table*\}} ... {\tt\string\end\{table*\}}, and \\
%\noindent{\tt\string\begin\{deluxetable*\}} ... {\tt\string\end\{deluxetable*\}}. \\
%
%\noindent This option is ignored in the onecolumn style.
%
%Some other style options are outlined in the commented sections of this 
%article.  Any combination of style options can be used.
%
%Two style options that are needed to fully use the new revision tracking
%feature, see Section \ref{sec:highlight}, are {\tt\string linenumbers} which 
%uses the lineno style file to number each article line in the left margin and 
%{\tt\string trackchanges} which controls the revision and commenting highlight
%output.
%
%There is also a new {\tt\string modern} option that uses a Daniel
%Foreman-Mackey and David Hogg design to produce stylish, single column
%output that has wider left and right margins. It is designed to have fewer
%words per line to improve reader retention. It also looks better on devices
%with smaller displays such as smart phones.

\section{Analysis} 
\label{sec:analysis}

Here we describe a few different approaches that have been used to deduce system parameters from observations of thermonuclear bursts.

\subsection{Inferring fuel composition} 
\label{subsec:fuelcomp}

\subsection{Observation-model comparisons} 
\label{subsec:lccompare}

The model-predicted quantities must then be converted to what a distant observer would see, by taking into account the effects of general relativistic time dilation and redshift.

Following \cite{lampe16}, we write the corrections as follows \cite[see also][]{keek11}. For self-consistency, we need to identify a mass and radius for the neutron star for which the Newtonian potential equals the GR potentials, i.e.
\begin{equation}
\frac{GM}{r^2} = \frac{GM_{\rm GR}}{R^2_{\rm GR}\sqrt{1-2GM_{\rm GR}/(c^2R_{\rm GR})}} = \frac{GM_{\rm GR}}{R^2_{\rm GR}}(1+z)
\end{equation}
where $1+z$ is the gravitational redshift:
\begin{equation}
1+z = \frac{1}{\sqrt{1-2GM_{\rm GR}/(c^2R_{\rm GR})}}
\end{equation}
This equality is generally achieved by assuming that $M=M_{\rm GR}$, and solving for $R_{\rm GR}$. 
We define $\xi$ such that $R_{\rm GR}=\xi R$. 
One advantage of this choice is that the mass accretion rate is identical in the Newtonian and observer frames, and also that $\xi = \sqrt{1 + z}$. 

An additional correction that must be applied is the effect of the anisotropic emission of the burst, based on the system inclination $i$. This effect has been simulated for a range of disk geometries \cite[e.g.][]{he16}. Because the emission region on the surface of the star is thought to be different for the burst and persistent emission, we define separate anisotropy parameters $\xi_b$ and $\xi_p$, with the sense that
\begin{equation}
L_{b,p} = 4\pi d^2\xi_{b,p}F_{b,p}
\end{equation}
where $L_b$, $L_p$ are the burst and persistent luminosity, $F_b$, $F_p$ are the corresponding (observed) fluxes, and $d$ is the source distance. With this sense it can be seen that values of $\xi_{b,p}<1$ correspond to emission preferentially beamed towards us, so that --- without the anisotropy corrections --- the isotropic luminosity inferred from the observed flux would overestimate the actual luminosity.

Of the model input parameters, we can also infer the persistent flux level expected given the model-assumed accretion rate $\dot{m}$, as follows: % gal03d, equation 2
\begin{equation}
F_{p,{\rm inf}} = \frac{L_p}{4\pi d^2} = 
% \frac{\dot{M} Q_{\rm grav}}{1+z} \left(\frac{R_{\rm NS}}{d}\right)^2 \xi_p^{-1}
% expression used in code
\frac{\dot{m} Q_{\rm grav}}{4\pi d^2 (1+z)\xi_p}% c_{\rm bol}}
\end{equation}
where $Q_{\rm grav} = c^2z/(1+z) \approx GM_{\rm NS}/R_{\rm NS}$ is the gravitational energy release per gram. % gal03d 

The predicted quantities can then be converted to values suitable for comparison to observations, as follows
\begin{eqnarray}
\Delta t_{{\rm pred},\infty} & = & (1+z)\Delta t_{\rm pred} \\
% E_{{\rm pred},\infty} & = & \xi^2E_{\rm pred}/(1+z)^2 \\ % CHECK
% add in the F_per (derived from the mdot) here; see Lampe et al. (2016), eq 8
t_{i,\infty} & = & (1+z)t_i \\
F_{{\rm pred},i,\infty} & = & \xi^2\frac{L_{{\rm pred},i}}{4\pi d^2\xi_b(1+z)^2}
\end{eqnarray}

For a given pair of observed and model lightcurves, the only parameters that affect the comparison are the source distance $d$, the anistotropy parameters $\xi_b$, $\xi_p$ (each a function of the inclination $i$), and the gravitational redshift $1+z$ (which also determines the parameter $\xi$).
%
We also introduce a ``nuisance'' parameter, $t_{\rm off}$, which is required to align the observed and predicted model lightcurve so as to minimise any residual differences. This parameter and the (time dilated) model timestamps $t_{i,\infty}$ are used to overlay the model predicted lightcurve onto the observed one.

Our approach is then to explore the parameter space of $(d, i, 1+z, t_{\rm off})$ to find the best set of parameters for which the comparison likelihood is maximised:
\begin{eqnarray}
\mathcal{L} & = & -f_F\left[\left( \frac{F_p-F_{p,\rm{inf}}}{\sigma_{p}}\right)^2 
    + \log\left(\frac{2\pi}{\sigma_p}\right)\right] \\
 & & -f_t\left[ \left( \frac{\Delta t-\Delta t_{{\rm pred},\infty}}{\sigma_t}\right)^2
    + \log\left(\frac{2\pi}{\sigma_t}\right)\right] + ...\\
\end{eqnarray}
where $f_F$...

Varying the redshift $1+z$ will allow us to obtain the best match between the model and observed lightcurve.
%this parameter cannot be varied arbitrarily, 
As each model run has been performed with a particular value of the surface gravity $g$, a particular value of $1+z$ implies in turn specific values of $M_{\rm NS}$ and $R_{\rm NS}$. 
%We take the approach of keeping $R_{\rm NS}$ fixed and allowing $M_{\rm NS}$ to vary to give the required change in $1+z$, since many equations of state have roughly constant radii over a range of masses \cite[e.g.][]{lp07}.

 
%\startlongtable
%\begin{deluxetable}{c|cc}
%\tablecaption{ApJ costs from 1991 to 2013\tablenotemark{a} \label{tab:table}}
%\tablehead{
%\colhead{Year} & \colhead{Subscription} & \colhead{Publication} \\
%\colhead{} & \colhead{cost} & \colhead{charges\tablenotemark{b}}\\
%\colhead{} & \colhead{(\$)} & \colhead{(\$/page)}
%}
%\colnumbers
%\startdata
%1991 & 600 & 100 \\
%1992 & 650 & 105 \\
%1993 & 550 & 103 \\
%1994 & 450 & 110 \\
%1995 & 410 & 112 \\
%1996 & 400 & 114 \\
%1997 & 525 & 115 \\
%1998 & 590 & 116 \\
%1999 & 575 & 115 \\
%2000 & 450 & 103 \\
%2001 & 490 &  90 \\
%2002 & 500 &  88 \\
%2003 & 450 &  90 \\
%2004 & 460 &  88 \\
%2005 & 440 &  79 \\
%2006 & 350 &  77 \\
%2007 & 325 &  70 \\
%2008 & 320 &  65 \\
%2009 & 190 &  68 \\
%2010 & 280 &  70 \\
%2011 & 275 &  68 \\
%2012 & 150 &  56 \\
%2013 & 140 &  55 \\
%\enddata
%\tablenotetext{a}{Adjusted for inflation}
%\tablenotetext{b}{Accounts for the change from page charges to digital quanta in April, 2011}
%\tablecomments{Note that {\tt \string \colnumbers} does not work with the 
%vertical line alignment token. If you want vertical lines in the headers you
%can not use this command at this time.}
%\end{deluxetable}


%\begin{deluxetable*}{ccCrlc}[b!]
%\tablecaption{Column math mode in an observation log \label{tab:mathmode}}
%\tablecolumns{6}
%\tablenum{2}
%\tablewidth{0pt}
%\tablehead{
%\colhead{UT start time\tablenotemark{a}} &
%\colhead{MJD start time\tablenotemark{a}} &
%\colhead{Seeing} & \colhead{Filter} & \colhead{Inst.} \\
%\colhead{(YYYY-mm-dd)} & \colhead{(d)} &
%\colhead{(arcsec)} & \colhead{} & \colhead{}
%}
%\startdata
%2012-03-26 & 56012.997 & \sim 0.\arcsec5 & H$\alpha$ & NOT \\
%2012-03-27 & 56013.944 & 1.\arcsec5 & grism & SMARTS \\
%2012-03-28 & 56014.984 & \nodata & F814M & HST \\
%2012-03-30 & 56016.978 & 1.\arcsec5\pm0.25 & B\&C & Bok \\
%\enddata
%\tablenotetext{a}{At exposure start.}
%\tablecomments{The ``C'' command column identifier in the 3 column turns on
%math mode for that specific column. One could do the same for the next
%column so that dollar signs would not be needed for H$\alpha$
%but then all the other text would also be in math mode and thus typeset
%in Latin Modern math and you will need to put it back to Roman by hand.
%Note that if you do change this column to math mode the dollar signs already
%present will not cause a problem. Table \ref{tab:mathmode} is published 
%in its entirety in the machine readable format.  A portion is
%shown here for guidance regarding its form and content.}
%\end{deluxetable*}


%%% Note that the \setcounter and \renewcommand are needed here because
%%% this example is using a mix of deluxetable and tabular.  Here the
%%% deluxetable counters are set with \tablenum but the situation is a bit
%%% more complex for tabular.  Use the first command to set the Table number
%%% to ONE LESS than it should be.  The next command will auto increment it
%%% to the desired number.
%\setcounter{table}{2}
%\begin{table}[h!]
%\renewcommand{\thetable}{\arabic{table}}
%\centering
%\caption{Decimal alignment made easy} \label{tab:decimal}
%\begin{tabular}{cD@{$\pm$}D}
%\tablewidth{0pt}
%\hline
%\hline
%Column & \multicolumn2c{Value} & \multicolumn2c{Uncertainty}\\
%\hline
%\decimals
%A & 1234     & 100.0     \\
%B &  123.4   &  10.1     \\
%C &  12.34   &   1.01    \\
%D &   1.234  &   0.101   \\
%E &    .1234 &   0.01001 \\
%F &   1.0    &    .      \\
%\hline
%\multicolumn{5}{c}{NOTE. - Two decimal aligned columns}
%\end{tabular}
%\end{table}

%%% The "ht!" tells LaTeX to put the figure "here" first, at the "top" next
%%% and to override the normal way of calculating a float position
%\begin{figure}[ht!]
%\plotone{cost.eps}
%\caption{The subscription and author publication costs from 1991 to 2013.
%The data comes from Table \ref{tab:table}.\label{fig:general}}
%\end{figure}

%\begin{equation}
%\bar v(p_2,\sigma_2)P_{-\tau}\hat a_1\hat a_2\cdots
%\hat a_nu(p_1,\sigma_1) ,
%\end{equation}

%\begin{eqnarray}
%\gamma^\mu  & = &
% \left(
%\begin{array}{cc}
%0 & \sigma^\mu_+ \\
%\sigma^\mu_- & 0
%\end{array}     \right) ,
% \gamma^5= \left(
%\begin{array}{cc}
%-1 &   0\\
%0 &   1
%\end{array}     \right)  , \\
%\sigma^\mu_{\pm}  & = &   ({\bf 1} ,\pm \sigma) , 
%\end{eqnarray}

%\begin{eqnarray}
%\hat a & = & \left(
%\begin{array}{cc}
%0 & (\hat a)_+\\
%(\hat a)_- & 0
%\end{array}\right), \nonumber \\
%(\hat a)_\pm & = & a_\mu\sigma^\mu_\pm 
%\end{eqnarray}

%% If you wish to include an acknowledgments section in your paper,
%% separate it off from the body of the text using the \acknowledgments
%% command.
\acknowledgments

To be added

%% To help institutions obtain information on the effectiveness of their 
%% telescopes the AAS Journals has created a group of keywords for telescope 
%% facilities.
%
%% Following the acknowledgments section, use the following syntax and the
%% \facility{} or \facilities{} macros to list the keywords of facilities used 
%% in the research for the paper.  Each keyword is check against the master 
%% list during copy editing.  Individual instruments can be provided in 
%% parentheses, after the keyword, but they are not verified.

\vspace{5mm}
\facilities{HST(STIS), Swift(XRT and UVOT), AAVSO, CTIO:1.3m,
CTIO:1.5m,CXO}

%% Similar to \facility{}, there is the optional \software command to allow 
%% authors a place to specify which programs were used during the creation of 
%% the manusscript. Authors should list each code and include either a
%% citation or url to the code inside ()s when available.

\software{astropy \citep{astropy13}  
%          Cloudy \citep{2013RMxAA..49..137F}, 
%          SExtractor \citep{1996A&AS..117..393B}
          }

%% Appendix material should be preceded with a single \appendix command.
%% There should be a \section command for each appendix. Mark appendix
%% subsections with the same markup you use in the main body of the paper.

%% Each Appendix (indicated with \section) will be lettered A, B, C, etc.
%% The equation counter will reset when it encounters the \appendix
%% command and will number appendix equations (A1), (A2), etc. The
%% Figure and Table counter will not reset.

%\appendix
%
%\section{Appendix information}

%% The reference list follows the main body and any appendices.
%% Use LaTeX's thebibliography environment to mark up your reference list.
%% Note \begin{thebibliography} is followed by an empty set of
%% curly braces.  If you forget this, LaTeX will generate the error
%% "Perhaps a missing \item?".
%%
%% thebibliography produces citations in the text using \bibitem-\cite
%% cross-referencing. Each reference is preceded by a
%% \bibitem command that defines in curly braces the KEY that corresponds
%% to the KEY in the \cite commands (see the first section above).
%% Make sure that you provide a unique KEY for every \bibitem or else the
%% paper will not LaTeX. The square brackets should contain
%% the citation text that LaTeX will insert in
%% place of the \cite commands.

%% We have used macros to produce journal name abbreviations.
%% \aastex provides a number of these for the more frequently-cited journals.
%% See the Author Guide for a list of them.

%% Note that the style of the \bibitem labels (in []) is slightly
%% different from previous examples.  The natbib system solves a host
%% of citation expression problems, but it is necessary to clearly
%% delimit the year from the author name used in the citation.
%% See the natbib documentation for more details and options.

\bibliography{all}
\bibliographystyle{apj}

%% This command is needed to show the entire author+affilation list when
%% the collaboration and author truncation commands are used.  It has to
%% go at the end of the manuscript.
%\allauthors

%% Include this line if you are using the \added, \replaced, \deleted
%% commands to see a summary list of all changes at the end of the article.
%\listofchanges

\end{document}

% End of file `sample61.tex'.
