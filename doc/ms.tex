%%
%% Beginning of file 'ms.tex'
%% Copied from sample61.tex by dkg 15.3.2018
%%
%% This is the paper describing the use and design of the concord suite of software
%% for burst observation-model comparisons
%%
%% See https://bitbucket.org/minbar/concord
%% and the companion jupyter notebook, Inferring burster properties.ipynb

%% AASTeX is now based on Alexey Vikhlinin's emulateapj.cls 
%% (Copyright 2000-2015).  See the classfile for details.

%% AASTeX requires revtex4-1.cls (http://publish.aps.org/revtex4/) and
%% other external packages (latexsym, graphicx, amssymb, longtable, and epsf).
%% All of these external packages should already be present in the modern TeX 
%% distributions.  If not they can also be obtained at www.ctan.org.

%% The first piece of markup in an AASTeX v6.x document is the \documentclass
%% command. LaTeX will ignore any data that comes before this command. The 
%% documentclass can take an optional argument to modify the output style.
%% The command below calls the preprint style  which will produce a tightly 
%% typeset, one-column, single-spaced document.  It is the default and thus
%% does not need to be explicitly stated.
%%
%%
%% using aastex version 6.3 from 2020 May; originally 6.1
\documentclass{aastex63}

%% The default is a single spaced, 10 point font, single spaced article.
%% There are 5 other style options available via an optional argument. They
%% can be envoked like this:
%%
%% \documentclass[argument]{aastex61}
%% 
%% where the arguement options are:
%%
%%  twocolumn   : two text columns, 10 point font, single spaced article.
%%                This is the most compact and represent the final published
%%                derived PDF copy of the accepted manuscript from the publisher
%%  manuscript  : one text column, 12 point font, double spaced article.
%%  preprint    : one text column, 12 point font, single spaced article.  
%%  preprint2   : two text columns, 12 point font, single spaced article.
%%  modern      : a stylish, single text column, 12 point font, article with
%% 		  wider left and right margins. This uses the Daniel
%% 		  Foreman-Mackey and David Hogg design.
%%
%% Note that you can submit to the AAS Journals in any of these 6 styles.
%%
%% There are other optional arguments one can envoke to allow other stylistic
%% actions. The available options are:
%%
%%  astrosymb    : Loads Astrosymb font and define \astrocommands. 
%%  tighten      : Makes baselineskip slightly smaller, only works with 
%%                 the twocolumn substyle.
%%  times        : uses times font instead of the default
%%  linenumbers  : turn on lineno package.
%%  trackchanges : required to see the revision mark up and print its output
%%  longauthor   : Do not use the more compressed footnote style (default) for 
%%                 the author/collaboration/affiliations. Instead print all
%%                 affiliation information after each name. Creates a much
%%                 long author list but may be desirable for short author papers
%%
%% these can be used in any combination, e.g.
%%
%% \documentclass[twocolumn,linenumbers,trackchanges]{aastex61}

\usepackage{amsmath}	% to enable cases environment

%% AASTeX v6.* now includes \hyperref support. While we have built in specific
%% defaults into the classfile you can manually override them with the
%% \hypersetup command. For example,
%%
%%\hypersetup{linkcolor=red,citecolor=green,filecolor=cyan,urlcolor=magenta}
%%
%% will change the color of the internal links to red, the links to the
%% bibliography to green, the file links to cyan, and the external links to
%% magenta. Additional information on \hyperref options can be found here:
%% https://www.tug.org/applications/hyperref/manual.html#x1-40003

%% If you want to create your own macros, you can do so
%% using \newcommand. Your macros should appear before
%% the \begin{document} command.
%%
\newcommand{\vdag}{(v)^\dagger}
\newcommand\aastex{AAS\TeX}
\newcommand\latex{La\TeX}

% Custom definitions

\newcommand{\eps}{{\rm erg\,s^{-1}}}
\newcommand{\epcs}{{\rm erg\,cm^{-2}\,s^{-1}}}
\newcommand{\epc}{{\rm erg\,cm^{-2}}}
\newcommand{\Xb}{\ensuremath{\overline{X}}}
\newcommand{\qnuc}{Q_{\rm nuc}}
\newcommand{\yign}{y_{\rm ign}}
\newcommand{\zcno}{Z_{\rm CNO}}
\newcommand{\xte}{{\it RXTE}}
\newcommand{\sax}{{\it BeppoSAX}}
\newcommand{\igr}{{\it INTEGRAL}}

\newcommand{\todo}[1]{\textbf{\textcolor{red}{TODO: #1}}} % Duncan

% standard redshift to use
% 1./sqrt(1.-2.*c.G*1.4*c.M_sun/((11.2*u.km).to('m')*c.c**2))            

\newcommand{\opz}{1.259}

%% Reintroduced the \received and \accepted commands from AASTeX v5.2

%\received{July 1, 2016}
%\revised{September 27, 2016}
%\accepted{\today}

%% Command to document which AAS Journal the manuscript was submitted to.
%% Adds "Submitted to " the arguement.
\submitjournal{ApJ}

%% Mark up commands to limit the number of authors on the front page.
%% Note that in AASTeX v6.1 a \collaboration call (see below) counts as
%% an author in this case.
%
%\AuthorCollaborationLimit=3
%
%% Will only show Schwarz, Muench and "the AAS Journals Data Scientist 
%% collaboration" on the front page of this example manuscript.
%%
%% Note that all of the author will be shown in the published article.
%% This feature is meant to be used prior to acceptance to make the
%% front end of a long author article more manageable. Please do not use
%% this functionality for manuscripts with less than 20 authors. Conversely,
%% please do use this when the number of authors exceeds 40.
%%
%% Use \allauthors at the manuscript end to show the full author list.
%% This command should only be used with \AuthorCollaborationLimit is used.

%% The following command can be used to set the latex table counters.  It
%% is needed in this document because it uses a mix of latex tabular and
%% AASTeX deluxetables.  In general it should not be needed.
%\setcounter{table}{1}

%%%%%%%%%%%%%%%%%%%%%%%%%%%%%%%%%%%%%%%%%%%%%%%%%%%%%%%%%%%%%%%%%%%%%%%%%%%%%%%%
%%
%% The following section outlines numerous optional output that
%% can be displayed in the front matter or as running meta-data.
%%
%% If you wish, you may supply running head information, although
%% this information may be modified by the editorial offices.
\shorttitle{Burst model-observation comparisons}
\shortauthors{Galloway et al.}
%%
%% You can add a light gray and diagonal water-mark to the first page 
%% with this command:
% \watermark{text}
%% where "text", e.g. DRAFT, is the text to appear.  If the text is 
%% long you can control the water-mark size with:
%  \setwatermarkfontsize{dimension}
%% where dimension is any recognized LaTeX dimension, e.g. pt, in, etc.
%%
%%%%%%%%%%%%%%%%%%%%%%%%%%%%%%%%%%%%%%%%%%%%%%%%%%%%%%%%%%%%%%%%%%%%%%%%%%%%%%%%

%% This is the end of the preamble.  Indicate the beginning of the
%% manuscript itself with \begin{document}.

\begin{document}

% \title{Thermonuclear burst model-observation comparisons with {\sc concord}}
\title{Robust inference of neutron-star parameters from thermonuclear burst 
% lightcurves}
observations}

%% LaTeX will automatically break titles if they run longer than
%% one line. However, you may use \\ to force a line break if
%% you desire. In v6.1 you can include a footnote in the title.

%% A significant change from earlier AASTEX versions is in the structure for 
%% calling author and affilations. The change was necessary to implement 
%% autoindexing of affilations which prior was a manual process that could 
%% easily be tedious in large author manuscripts.
%%
%% The \author command is the same as before except it now takes an optional
%% arguement which is the 16 digit ORCID. The syntax is:
%% \author[xxxx-xxxx-xxxx-xxxx]{Author Name}
%%
%% This will hyperlink the author name to the author's ORCID page. Note that
%% during compilation, LaTeX will do some limited checking of the format of
%% the ID to make sure it is valid.
%%
%% Use \affiliation for affiliation information. The old \affil is now aliased
%% to \affiliation. AASTeX v6.1 will automatically index these in the header.
%% When a duplicate is found its index will be the same as its previous entry.
%%
%% Note that \altaffilmark and \altaffiltext have been removed and thus 
%% can not be used to document secondary affiliations. If they are used latex
%% will issue a specific error message and quit. Please use multiple 
%% \affiliation calls for to document more than one affiliation.
%%
%% The new \altaffiliation can be used to indicate some secondary information
%% such as fellowships. This command produces a non-numeric footnote that is
%% set away from the numeric \affiliation footnotes.  NOTE that if an
%% \altaffiliation command is used it must come BEFORE the \affiliation call,
%% right after the \author command, in order to place the footnotes in
%% the proper location.
%%
%% Use \email to set provide email addresses. Each \email will appear on its
%% own line so you can put multiple email address in one \email call. A new
%% \correspondingauthor command is available in V6.1 to identify the
%% corresponding author of the manuscript. It is the author's responsibility
%% to make sure this name is also in the author list.
%%
%% While authors can be grouped inside the same \author and \affiliation
%% commands it is better to have a single author for each. This allows for
%% one to exploit all the new benefits and should make book-keeping easier.
%%
%% If done correctly the peer review system will be able to
%% automatically put the author and affiliation information from the manuscript
%% and save the corresponding author the trouble of entering it by hand.

\correspondingauthor{Duncan K. Galloway}
\email{duncan.galloway@monash.edu}

\author[0000-0002-6558-5121]{Duncan K. Galloway}

\author{Zac Johnston}
\author{Adelle Goodwin}

\affil{School of Physics \& Astronomy \\
Monash University \\
Clayton, VIC 3800, Australia}

%\author{August Muench}
%\affiliation{American Astronomical Society \\
%2000 Florida Ave., NW, Suite 300 \\
%Washington, DC 20009-1231, USA}
%\collaboration{(AAS Journals Data Scientists collaboration)}
%
%\author{Butler Burton}
%\affiliation{National Radio Astronomy Observatory}
%\affiliation{AAS Journals Associate Editor-in-Chief}
%\nocollaboration
%
%\author{Amy Hendrickson}
%\altaffiliation{Creator of AASTeX v6.1}
%\affiliation{TeXnology Inc.}
%\collaboration{(LaTeX collaboration)}
%
%\author{Julie Steffen}
%\affiliation{AAS Director of Publishing}
%\affiliation{American Astronomical Society \\
%2000 Florida Ave., NW, Suite 300 \\
%Washington, DC 20009-1231, USA}
%
%\author{Jeff Lewandowski}
%\affiliation{IOP Senior Publisher for the AAS Journals}
%\affiliation{IOP Publishing, Washington, DC 20005}

%% Note that the \and command from previous versions of AASTeX is now
%% depreciated in this version as it is no longer necessary. AASTeX 
%% automatically takes care of all commas and "and"s between authors names.

%% AASTeX 6.1 has the new \collaboration and \nocollaboration commands to
%% provide the collaboration status of a group of authors. These commands 
%% can be used either before or after the list of corresponding authors. The
%% argument for \collaboration is the collaboration identifier. Authors are
%% encouraged to surround collaboration identifiers with ()s. The 
%% \nocollaboration command takes no argument and exists to indicate that
%% the nearby authors are not part of surrounding collaborations.

%% Mark off the abstract in the ``abstract'' environment. 
\begin{abstract}

% AAS Journals, the Astrophysical Journal (ApJ), the
%Astrophysical Journal Letters (ApJL), and Astronomical Journal (AJ), all
%have a 250 word limit for the abstract.  If you exceed this length the
%Editorial office will ask you to shorten it.

Thermonuclear (type-I) bursts arise from unstable ignition of accumulated fuel on the surface of neutron stars in low-mass X-ray binaries. Measurements of burst properties enable observers to infer the properties of the host neutron stars, but a number of confounding astrophysical effects contribute to systematic uncertainties. 
%
In particular, comparison of observed burst properties with the predictions of numerical models is a 
% key factor in determining the properties of the host neutron star.
powerful method of constraining host properties.
%
Here we describe some commonly-used approaches for determining system parameters, including composition of the burst fuel, and introduce a new suite of software tools, {\sc concord}, intended to make comprehensive model-observation comparisons practical.

\end{abstract}

%% Keywords should appear after the \end{abstract} command. 
%% See the online documentation for the full list of available subject
%% keywords and the rules for their use.
\keywords{editorials, notices --- 
miscellaneous --- catalogs --- surveys}

% ------------------------------------------------------------------------------
%% From the front matter, we move on to the body of the paper.
%% Sections are demarcated by \section and \subsection, respectively.
%% Observe the use of the LaTeX \label
%% command after the \subsection to give a symbolic KEY to the
%% subsection for cross-referencing in a \ref command.
%% You can use LaTeX's \ref and \label commands to keep track of
%% cross-references to sections, equations, tables, and figures.
%% That way, if you change the order of any elements, LaTeX will
%% automatically renumber them.

%% We recommend that authors also use the natbib \citep
%% and \citet commands to identify citations.  The citations are
%% tied to the reference list via symbolic KEYs. The KEY corresponds
%% to the KEY in the \bibitem in the reference list below. 

\section{Introduction} \label{sec:intro}

Thermonuclear (type-I) bursts are observed from neutron stars accreting from low-mass ($\lesssim 1\ M_\odot$) binary companions \cite[e.g.][]{gal17b}. ...

Here we explore the ways that host system  parameters affect the observed properties of thermonuclear bursts, and assess to what degree these effects can be corrected for.
%
In \S\ref{sec:data} we describe the typical characteristics of burst observations along with those properties that can be 
% predicted by numerical models.
inferred from simple calculations or by comparison with numerical models.
% 
In \S\ref{sec:analysis} we explore a number of methods to infer properties of the bursting source, based on the extent of available observational data, and assess their accuracy.
%
In \S\ref{sec:simulations} we apply these methods to a range of simulated data to assess what degree of systematic errors these measurements might be subject to.

%\latex\ \footnote{\url{http://www.latex-project.org/}} is a document markup
%language that is particularly well suited for the publication of
%mathematical and scientific articles \citep{lamport94}. \latex\ was written
%in 1985 by Leslie Lamport who based it on the \TeX\ typesetting language
%which itself was created by Donald E. Knuth in 1978.  In 1988 a suite of
%\latex\ macros were developed to investigate electronic submission and
%publication of AAS Journal articles \citep{1989BAAS...21..780H}.  Shortly
%afterwards, Chris Biemesdefer merged these macros and more into a \latex\
%2.08 style file called \aastex.  These early \aastex\ versions introduced
%many common commands and practices that authors take for granted today.
%Substantial revisions
%were made by Lee Brotzman and Pierre Landau when the package was updated to
%v4.0.  AASTeX v5.0, written in 1995 by Arthur Ogawa, upgraded to \latex\ 2e
%which uses the document class in lieu of a style file.  Other improvements
%to version 5 included hypertext support, landscape deluxetables and
%improved figure support to facilitate electronic submission.  
%\aastex\ v5.2 was released in 2005 and introduced additional graphics
%support plus new mark up to identifier astronomical objects, datasets and
%facilities.
%
%In 1996 Maxim Markevitch modified the AAS preprint style file, aaspp4.sty,
%to closely emulate the very tight, two column style of a typeset
%Astrophysical Journal article.  The result was emulateapj.sty.  A year
%later Alexey Vikhlinin took over development and maintenance.  In 2001 he
%converted emulateapj into a class file in \latex\ 2e and in 2003 Vikhlinin
%completely rewrote emulateapj based on the APS Journal's RevTEX class.
%
%During this time emulateapj gained growing acceptance in the astronomical
%community as it filled an author need to obtain an approximate number of
%manuscript pages prior to submission for cost and length estimates. The
%tighter typeset also had the added advantage of saving paper when printing
%out hard copies.
%
%Even though author publication charges are no longer based on print pages
%\footnote{see Section \ref{sec:pubcharge} in the Appendix for more details
%about how current article costs are calculated.} the emulateapj class file
%has proven to be extremely popular with AAS Journal authors.  An informal
%analysis of submitted \latex\ manuscripts in 2015 revealed that $\sim$65\%
%either called emulateapj or have a commented emulateapj classfile call
%indicating it was used at some stage of the manuscript construction.
%Clearly authors want to have access to a tightly typeset version of the
%article when corresponding with co-authors and for preprint submissions.
%
%When planning the next \aastex\ release the popularity of emulateapj played
%an important roll in the decision to drop the old base code and adopt and
%modify emulateapj for \aastex\ v6.+ instead.  The change brings \aastex\
%inline with what the majority of authors are already using while still
%delivering new and improved features.  \aastex\ v6.0 and v6.1 were written
%by Amy Hendrickson and released in January and October 2016, respectively.
%Some of the new features in v6.0 included:
%
%\begin{enumerate}
%\item improved citations for third party data repositories and software,
%\item easier construction of matrix figures consisting of multiple 
%encapsulated postscript (EPS) or portable document format (PDF) files,
%\item figure set mark up for large collections of similar figures,
%\item color mark up to easily enable/disable revised text highlighting,
%\item improved url support, and
%\item numerous table options such as the ability to hide columns, column 
%decimal alignment, automatic column math mode and numbering, plus splitting of
%wide tables.
%\end{enumerate}
%
%The new features in v6.1 are:
%
%\begin{enumerate}
%\item ORCID support for preprints,
%\item improved author, affiliation and collaboration mark up,
%\item reintroduced the old AASTeX v5.2 {\tt\string\received}, 
%      {\tt\string\revised}, {\tt\string\accepted}, and
%      {\tt\string\published} commands plus
%      added the new {\tt\string\submitjournal} command to document
%      which AAS Journal the manuscript was submitted to, plus
%\item new typeset style options.
%\end{enumerate}
%
%The rest of this article provides information and examples on how to create
%your own AAS Journal manuscript with v6.1.  Special emphasis is placed on
%how to use the full potential of \aastex\ v6+.  The next section describes
%the different manuscript styles available and how they differ from past
%releases.  Section \ref{sec:floats} describes how tables and figures are
%placed in a \latex\ document. Specific examples of tables, Section
%\ref{subsec:tables}, and figures, Section \ref{subsec:figures}, are also
%provided.  Section \ref{sec:displaymath} discusses how to display math and
%incorporate equations in a manuscript while Section \ref{sec:highlight}
%discuss how to use the new revision mark up.  The last section,
%\ref{sec:cite}, shows how recognize software and external data as first
%class references in the manuscript bibliography.  An appendix is included
%to show how to construct one and provide some information on how article
%charges are calculated.  Additional information is available both embedded
%in the comments of this \latex\ file and in the online documentation at
%\url{http://journals.aas.org/authors/aastex.html}.

% ------------------------------------------------------------------------------
\section{Observations and modelling} 
\label{sec:data}

%\subsection{Burst observations}
%\label{subsec:data}

The fundamental observable that we measure is the time-history of the burst flux, the lightcurve $(t_i,F_{i})$. Here the $F_{i}$ may be in instrumental units of count~s$^{-1}$ (or perhaps count~s$^{-1}$~cm$^{-2}$), or may be estimates of the bolometric flux at time $t_i$, derived from model fits of the spectrum at time $t_i$.
%
Provided the $F_i$ are in units of flux, or the flux can be estimated from the intensity,  we may measure the peak flux, $F_{\rm pk}$, and integrate to estimate the burst fluence, $E_b$. 

Given one or more bursts, we can derive constraints on the recurrence time $\Delta t$. A pair of bursts separated by an interval with complete coverage by X-ray instruments offer an unambiguous measurement of the recurrence time, but such measurements are rare due to the typically low duty cycle for most instruments \cite[e.g.][]{minbar}.
%
Such cases also tend to be biased towards lower recurrence times, which may be dominated by so-called ``short waiting time'' bursts, which likely arise from ignition of unburnt fuel from the previous event \cite[]{keek10}. The recurrence time for such events likely does not represent the actual time required to meet the ignition conditions at the base of the fuel layer, and so are generally excluded for the purposes of inferring properties of the burst sources.

Where one or more bursts $N$ are observed in low-duty cycle observations, the rate can be estimated based on the total exposure, i.e.
\begin{equation}
\langle R\rangle = \frac{N}{\sum^n T_i}
\end{equation}
where each of $n$ observartions has exposure $T_i$. The probable range of the average rate can be estimated assuming Poisson counting statistics for the uncertainty on the burst number $N$. Because of the approximately periodic behavior common for bursts, and the similarity between low-Earth orbital periods and the burst recurrence times, such uncertainty estimates may be wildly incorrect.

From data taken prior to the burst ignition we can measure the persistent flux $F_{\rm per}$, a measure of the accretion rate; 
% and provided more than one burst is observed, we may also measure (or estimate) the recurrence time, $\Delta t$. 
%
these parameters may be combined to estimate the $\alpha$-value, the (observed) ratio of the burst to persistent flux: 
\begin{equation}
\alpha = \frac{\Delta t F_{\rm per} c_{\rm bol}}{E_B}
\end{equation}
where $c_{\rm bol}$ is the bolometric correction giving the inverse fraction of total persistent flux emitted in the instrumental band.

additional data analysis comments
\begin{itemize}
\item correcting lightcurve for variations e.g. from dips, or achromatic variations
\item using different spectral models than blackbody
\item allowing the persistent flux to vary
\item maybe a table with the parameters?
\end{itemize}

\subsection{Burster distances}
\label{subsec:distance}

A significant fraction of observed bursts exhibit characteristic variations in their spectral hardness (or the blackbody temperature and normalisation) around their maximum that indicate the presence of photospheric radius-expansion (PRE). 
%
Even if such variations are not present, 
%
the peak flux $F_{\rm pk}$ provides constraints on the source distance, since the maximum burst luminosity is limited to (roughly) the Eddington luminosity \cite[e.g.][]{lew93}:
% copied from bcatalog, and updated prefac
% (8.*pi*const.G*const.m_p*1.4*const.M_sun*const.c/(const.sigma_T*1.259)).to('erg s-1')                                                         
% Out[16]: <Quantity 2.79569695e+38 erg / s>
\begin{eqnarray}
  L_{\rm Edd,\infty} & = & \frac{8\pi G m_p M_{\rm NS} c
  [1+(\alpha_{\rm T}T_{\rm e})^{0.86}]} {\sigma_T(1+X)[1+z(R)]} % \xi_b
       \nonumber \\
%  & = & 2.7\times10^{38} \left(\frac{M_{\rm NS}}{1.4M_\odot}\right)
  & = & 2.80\times10^{38} \left(\frac{M_{\rm NS}}{1.4M_\odot}\right)
 \frac{1+(\alpha_{\rm T}T_{\rm e})^{0.86}}{(1+X)}
\nonumber \\ & & \times\  % line break
%    \left[\frac{1+z(R)}{1.31}\right]^{-1}\ % \xi_b\
    \left[\frac{1+z(R)}{\opz}\right]^{-1}\ % \xi_b\
              \eps
  \label{ledd}
\end{eqnarray}
Here
$T_{\rm e}$ is the
effective temperature of the atmosphere, $\alpha_{\rm T}$ parametrizes the temperature dependence of the electron scattering
opacity \cite[$\simeq 2.2\times10^{-9}$~K$^{-1}$;][]{lew93},
$m_p$ is the proton mass, $\sigma_T$ the Thompson 
cross-section, and $X$ the
mass fraction of hydrogen in the atmosphere.
% ($\approx0.7$ for cosmic abundances).
The final factor in
square brackets represents the gravitational redshift 
at the photosphere $1+z(R)=(1-2GM_{\rm NS}/R
c^2)^{-1/2}=1.259$ for $M_{\rm NS}=1.4\ M_\odot$ and $R=R_{\rm NS}=11.2$~km.
%
The effective redshift measured at the peak of a PRE burst may be lower than the value at the NS surface, while the photosphere is expanded during the radius expansion episode (i.e. $R\ge R_{\rm NS}$).

% {\it Mention also the empirical value here... } actually later

By equating the maximum flux of these events with the Eddington luminosity, the distance to the source can be estimated as
% p is the prefactor from the L_Edd expression above
% np.sqrt(p/(4.*pi*3e-8*u.erg/u.cm**2/u.s)).to('kpc')                     
% Out[26]: <Quantity 8.82528911 kpc>
\begin{eqnarray}
 d & = & \left(\frac{L_{\rm Edd,\infty}}{4\pi \xi_b F_{\rm pk, RE}}\right)^{1/2} \nonumber \\
%   & = & 8.6
   & = & 8.83
	\left( \frac{\xi_b F_{\rm pk, RE}}{3\times10^{-8}\ \epcs} \right)^{-1/2}
       	\left(\frac{M_{\rm NS}}{1.4M_\odot}\right)^{1/2}
\nonumber \\ & & \times\  % line break
%	\left[\frac{1+z(R)}{1.31}\right]^{-1/2}
	\left[\frac{1+z(R)}{\opz}\right]^{-1/2}
	(1+X)^{-1/2}\ {\rm kpc}
 \label{disteq}
\end{eqnarray}
% 
Here the $\xi_b$ factor accounts for the possible anisotropy of the burst emission, 
which is discussed in the next section.

As is obvious from equation \ref{disteq}, several factors will contribute to statistical (and possibly systematic) uncertainties in distance estimates derived in this manner. First, the neutron star mass $M_{\rm NS}$ is unknown...
%
Second, the radius $R$ at which the redshift should be calculated may not be clear from the observations. In principle one could take the inferred blackbody normalisation at the time of maximum flux, but for most PRE bursts the maximum flux is achieved close to the ``touchdown'' point, where the photosphere has returned (more or less) to the neutron star surface \cite[e.g.][]{gal06a}.
%
Third, the possible range/likely value for $X$...
%
Fourth, even if the inclination $i$ of the system is known, converting to the anisotropy factor requires modelling of the effect of the disk on the radiation field arising from the burst \cite[e.g.][]{he16}.

These issues may be avoided by adopting instead the empirical Eddington luminosity measured by \cite[]{kuul03a}...

\subsection{Emission anisotropy}
\label{subsec:anisotropy}

The anisotropy factor $\xi_b$ introduced into the expression for the distance (equation  \ref{disteq}) is defined in the same sense as \citep{fuji88}, such that the total luminosity of the burst $L_b=4\pi d^2\xi_b F_{\rm pk}$. The range of values of $\xi_b$ can thus be understood that if $\xi_b<1$ the burst flux is enhanced (i.e. preferentially beamed) toward our line of sight (so that $L_b$ would be {\it overestimated} were it not included in the calculation), while if $\xi_b>1$ the burst flux is suppressed.
%
Since the distribution of the burst flux over the neutron star is likely different from that of the persistent emission, we define a different anisotropy factor $\xi_p$ relating the observed persistent flux $F_p$ to the total luminosity. Modelling \cite[]{he16}...

{\it describe how we treat this }

\subsection{Burst energetics}
\label{subsec:energetics}

The fluence (and the shape of the burst profile) is determined by the amount of accumulated fuel, and its composition. For accretion of mixed H/He with hydrogen mass fraction $X_0$, the composition at ignition may be modified substantially by $\beta$-limited CNO burning. In extreme cases the recurrence time is sufficiently long that the accreted hydrogen at the base of the fuel layer is exhausted, and ignition of intense, fast He-rich bursts occur.

For a fuel layer consisting of mixed H/He, $\qnuc$ depends on the mean hydrogen fraction at ignition, \Xb:
% The nuclear energy generation rate $\qnuc$ depends on the composition; 
studies with the 1-D numerical code {\sc kepler} suggest that 
\begin{eqnarray}
Q_{\rm nuc} &= & 1.31+6.95\Xb-1.92\Xb^2 \nonumber \\
& \approx & 1.35 + 6.05 \Xb
\label{eq:qnuc}
\end{eqnarray}
\cite[]{goodwin19a}.
% where $\Xb$ is the average H-fraction of the fuel layer at ignition. 
For fuel with solar composition (i.e. $X=0.7$), $Q_{\rm nuc}=5.22$~MeV/nucleon.

{\it Perhaps put in here some estimates of the fuel composition as a function of $X_0$ and recurrence time; could perhaps develop a ``prior'' for $\Xb$ }

Provided an estimate of $\Xb$ is available, we can estimate the burst column that has ignited as
% (1e-6*u.erg/u.cm**2*(10.*u.kpc).to('cm')**2/(5.22*u.MeV/const.m_p)*1.259
%     ...: /(11.2*u.km)**2).to('g cm-2')
% Out[36]: <Quantity 1.9112067e+08 g / cm2>
\begin{eqnarray}
y & = & \frac{L_b d^2 (1+z)}{R_{\rm NS}^2 Q_{\rm nuc}}\ 
                                                            \nonumber \\
%  & = & 3.0\times10^8 \left(\frac{\xi_b E_{\rm b}}{10^{-6}\ \epc}\right)
  & = & 1.91\times10^8 \left(\frac{\xi_b E_{\rm b}}{10^{-6}\ \epc}\right)
                      \left(\frac{d}{10\ {\rm kpc}}\right)^2\
%             \left(\frac{Q_{\rm nuc}}{4.4\ {\rm MeV/nucleon}}\right)^{-1}
             \left(\frac{Q_{\rm nuc}}{5.22\ {\rm MeV/nucleon}}\right)^{-1}
\nonumber \\ & & \times\  % line break
%                      \left(\frac{1+z}{1.31}\right)
%                      \left(\frac{R_{\rm NS}}{10\ {\rm km}}\right)^{-2}
                      \left(\frac{1+z}{\opz}\right)
                      \left(\frac{R_{\rm NS}}{11.2\ {\rm km}}\right)^{-2}
                      {\rm g\,cm^{-2}}
\label{eq:column}
\end{eqnarray}
% {\it need to update prefactor, denominator in $Q_{\rm nuc}$ term here } - done

\subsection{Burst models}
\label{subsec:models}

Thermonuclear burst simulations \cite[e.g. with {\sc kepler};][]{woos04} are usually carried out in a plane-parallel grid with constant gravity, $g$. The principal input parameters include the accretion rate (per unit area), $\dot{m}$; the composition of the accreted fuel, usually quantified as $(X_0, Z_{\rm CNO})$ where $X_0$ is the mass fraction of hydrogen, and $Z_{\rm CNO}$ the mass fraction of CNO nuclei, which drive the hot-CNO cycle burning between bursts; and a parameter describing the degree of heating from below the model domain, usually labeled as ``base flux'' $Q_b$.

The model predictions include a recurrence time $\Delta t_{\rm pred}$, burst energy $E_{\rm pred}$ and lightcurve $(t_i,L_{{\rm pred},i})$ covering the extent of the burst.
%
The accretion rate $\dot{m}$ may also be converted to a persistent flux level for comparison with observations.

\subsection{Simulated observations}
\label{subsec:simobs}

In order to test the analysis approaches described in this paper, we generated sets of simulated data based on the {\sc kepler} models described in \S\ref{subsec:models}. 
%
This procedure is identical to that used to convert model predictions to data for direct comparison to observations, as described in \S\ref{subsec:lccompare}.
%
We seek to calculate the burst lightcurve that would be observed given a model lightcurve and neutron star parameters including distance $d$; inclination $i$; and surface redshift $(1+z)_{\rm fit}$ (so labeled to distinguish from the corresponding value implict to the model). The process can be summarised as follows. [{\bf note doesn't include any steps to deal with the redshift inconsistency, see below}]
\begin{enumerate}
\item Multiply the time bins for the predicted burst lightcurve by the trial gravitational redshift $(1+z)_{\rm fit}$, thereby ``stretching'' the profile to account for the general relativistic time dilation at the neutron-star surface
% \item Shift the predicted burst lightcurve by $t_{\rm off}$
\item Apply the same correction to the model-predicted recurrence time $\Delta t_{\rm pred}$.
\item Interpolate the model-predicted lightcurve onto a set of observational time bins, corresponding (for example) to the typical resolution for time-resolved spectroscopy (0.25~s)
\item Translate the model-predicted luminosity to the corresponding quantity measured by a distant observer, by 
% multiplying by $[\xi/(1+z)]^2$
dividing by $(1+z)_{\rm fit}$
\item Convert the luminosity to (isotropic) flux by dividing by the distance factor, $4\pi d^2$ \label{distfac}
\item Take into account the expected anisotropy effects due to the system inclination, by dividing the luminosity by the anisotropy factor $\xi_b$ \label{anisofac}
\item Calculate the persistent flux expected for the model-assumed accretion rate $\dot{m}$, and apply the same corrections as for the burst flux in steps \ref{distfac} and \ref{anisofac} (adopting a separate anisotropy factor $\xi_p$ appropriate for the persistent flux). We also divide by a bolometric correction factor $c_{\rm bol}$ accounting for the limited instrumental passband.
\end{enumerate}
% Below we describe each of these steps in more detail

We determine the general relativistic (GR) corrections within the constraints of the numerical models, which are typically calculated assuming a Newtonian potential with gravity $g=GM/R^2$, where $M$ and $R$ are the equivalent Newtonian mass and radius.
%
We are free (in principle) to vary the surface redshift $1+z$, for example to achieve improved agreement with an observed profile (although see below). 

The model time bins  $t_i$ and the predicted burst recurrence time $\Delta t_{\rm pred}$ are converted to values as would be measured by a distant observer as follows:
\begin{eqnarray}
t_{i,\infty} & = & (1+z)t_i \\
\Delta t_{{\rm pred},\infty} & = & (1+z)\Delta t_{\rm pred}
\end{eqnarray}

Care must be taken to ensure the Newtonian model predictions can be correctly translated to include the GR corrections expected for quantities at the neutron star surface.
%
Following \cite{lampe16}, 
% we write the corrections as follows \cite[see also][]{keek11}. For self-consistency, 
we identify a mass and radius for the neutron star for which the Newtonian potential equals the GR potentials, i.e.
\begin{equation}
\frac{GM}{R^2} = \frac{GM_{\rm GR}}{R^2_{\rm GR}\sqrt{1-2GM_{\rm GR}/(c^2R_{\rm GR})}} = \frac{GM_{\rm GR}}{R^2_{\rm GR}}(1+z)
\end{equation}
where $1+z$ is the gravitational redshift:
\begin{equation}
1+z = \frac{1}{\sqrt{1-2GM_{\rm GR}/(c^2R_{\rm GR})}}
\end{equation}
This equality is generally achieved by assuming that $M=M_{\rm GR}$, and solving for $R_{\rm GR}$. 
We define $\xi$ such that $R_{\rm GR}=\xi R$. 
One advantage of this choice is that the mass accretion rate is identical in the Newtonian and observer frames, and also that $\xi = \sqrt{1 + z}$. In that case, the model-predicted luminosity is related to the luminosity measured by a distant observer, by 
\begin{eqnarray}
L_\infty & = & \xi^2L/(1+z)^2 \nonumber \\
& = & L/(1+z)
\end{eqnarray}

% The corrections described below
However, since the model is to be transformed by a different value of $(1+z)_{\rm fit}$, we must take care to consistently transform the model parameters in accordance with this choice.
%
We can achieve this objective as follows. The surface gravity $g$ adopted for the model, and the trial redshift $(1+z)_{\rm fit}$ imply a different neutron star mass and radius than adopted for the model: 
%
% We note that the combination of surface gravity $g$ (assumed for the model) and adopted $1+z$  uniquely specifies the neutron star mass $M_{\rm NS}$ and radius $R_{\rm NS}$:
\begin{eqnarray}
% as used in function calc_mr in burstclass.py
R_{\rm NS,fit} &=& c^2\frac{(1+z)_{\rm fit}^2-1}{2g(1+z)_{\rm fit}}\\
M_{\rm NS,fit} &=& \frac{gR_{\rm NS,fit}^2}{G(1+z)_{\rm fit}}
\end{eqnarray}
Thus, by identifying the optimal value of $(1+z)_{\rm fit}$ for comparison to a particular observation, we can constrain the mass and radius, at a fixed $g$. [{\bf so which one should I use? and should I make any other changes to the recipe?}]

{\it This likely duplicates earlier discussion} \\
%
An additional correction that must be applied is the effect of the anisotropic emission of the burst, based on the system inclination $i$. This effect has been simulated for a range of disk geometries \cite[e.g.][]{he16}. Because the emission region on the surface of the star is thought to be different for the burst and persistent emission, we define separate anisotropy parameters $\xi_b$ and $\xi_p$, with the sense that
\begin{equation}
L_{b,p} = 4\pi d^2\xi_{b,p}F_{b,p}
\end{equation}
where $L_b$, $L_p$ are the burst and persistent luminosity, $F_b$, $F_p$ are the corresponding (observed) fluxes, and $d$ is the source distance. With this sense it can be seen that values of $\xi_{b,p}<1$ correspond to emission preferentially beamed towards us, so that --- without the anisotropy corrections --- the isotropic luminosity inferred from the observed flux would overestimate the actual luminosity.

Of the model input parameters, we can also infer the persistent flux level expected given the model-assumed accretion rate $\dot{m}$, as follows: % gal03d, equation 2
\begin{equation}
% function fper in burstclass.py
F_{p,\infty} = \frac{L_p}{4\pi d^2} = 
% \frac{\dot{M} Q_{\rm grav}}{1+z} \left(\frac{R_{\rm NS}}{d}\right)^2 \xi_p^{-1}
% expression used in code
\frac{\dot{m} Q_{\rm grav}}{4\pi d^2 (1+z)\xi_p c_{\rm bol}}
\end{equation}
% Q_grav already defined in the previous section
% where $Q_{\rm grav} = c^2z/(1+z) \approx GM_{\rm NS}/R_{\rm NS}$ is the gravitational energy release per gram. % gal03d 
%
The bolometric correction $c_{\rm bol}$ accounts for the expermental limitation that the persistent flux can only be measured over a limited instrumental passband. The bolometric correction is the ratio of the estimated bolometric flux to the band-limited value.

We performed simulations based on {\sc kepler} simulations drawn from our range of in-hand simulations. The range of input values are listed in Table \ref{tab:simparams}.

\begin{deluxetable}{lccc}
\tablecaption{Parameter ranges for simulated data \label{tab:simparams}}
\tablehead{
\colhead{Parameter} & \colhead{Symbol} & \colhead{Units} & \colhead{Range} }
\startdata
Surface gravity & $g$ & $10^{14}\ {\rm g\,cm^{-2}}$ & \\
Accreted hydrogen mass fraction & $X_0$ & \nodata & \\
Metallicity & $Z_{\rm CNO}$ & \nodata & \\
Base flux & $Q_b$ & MeV/nucleon & \\
%
% samples are drawn randomly over the ranges below, which might be tightened for the 
% final production versions. In particular I don't know if such high redshifts are 
% physical
Distance & $d$ & kpc & 4--12\\
System inclination\tablenotemark{a} & $i$ & degrees & 0--75\\
Surface redshift & $1+z$ & \nodata & 1--2\\
\enddata
\tablenotetext{a}{Drawn from a random distribution uniform in $\cos i$}
% \tablecomments{Note that {\tt \string \colnumbers} does not work with the 
% vertical line alignment token. If you want vertical lines in the headers you
% can not use this command at this time.}
\end{deluxetable}


%The predicted quantities can then be converted to values suitable for comparison to observations, as follows
%\begin{equation}
%F_{{\rm pred},i,\infty}  =  \xi^2\frac{L_{{\rm pred},i}}{4\pi d^2\xi_b(1+z)^2}
%\end{equation}


%The default style in \aastex\ v6.1 is a tight single column style, e.g.  10
%point font, single spaced.  The single column style is very useful for
%article with wide equations. It is also the easiest to style to work with
%since figures and tables, see Section \ref{sec:floats}, will span the
%entire page, reducing the need for address float sizing.
%
%To invoke a two column style similar to the what is produced in
%the published PDF copy use \\
%
%\noindent {\tt\string\documentclass[twocolumn]\{aastex61\}}. \\
%
%\noindent Note that in the two column style figures and tables will only
%span one column unless specifically ordered across both with the ``*'' flag,
%e.g. \\
%
%\noindent{\tt\string\begin\{figure*\}} ... {\tt\string\end\{figure*\}}, \\
%\noindent{\tt\string\begin\{table*\}} ... {\tt\string\end\{table*\}}, and \\
%\noindent{\tt\string\begin\{deluxetable*\}} ... {\tt\string\end\{deluxetable*\}}. \\
%
%\noindent This option is ignored in the onecolumn style.
%
%Some other style options are outlined in the commented sections of this 
%article.  Any combination of style options can be used.
%
%Two style options that are needed to fully use the new revision tracking
%feature, see Section \ref{sec:highlight}, are {\tt\string linenumbers} which 
%uses the lineno style file to number each article line in the left margin and 
%{\tt\string trackchanges} which controls the revision and commenting highlight
%output.
%
%There is also a new {\tt\string modern} option that uses a Daniel
%Foreman-Mackey and David Hogg design to produce stylish, single column
%output that has wider left and right margins. It is designed to have fewer
%words per line to improve reader retention. It also looks better on devices
%with smaller displays such as smart phones.

\section{Analysis} 
\label{sec:analysis}

Here we describe different approaches that have been used to deduce system parameters from observations of thermonuclear bursts.
%
We focus on adopting a suitable approach given the availability of data, which will differ for different sources

\subsection{Zero bursts -- constraining the distance}
\label{subsec:zerobursts}

While it may seem strange to base an analysis on the {\it non-}detection of bursts, practically even for the best-studied sources the X-ray observation duty cycle is of order a few percent. There is a high probability that bursts will be missed, particularly if the accretion rate is low (and hence the bursts are infrequent). In extreme cases this can mean that no bursts whatsoever are observed, which may instead be explained by the compact object being a black hole rather than a neutron star.

In cases where we can be confident that the compact object is a neutron star (e.g. where persistent pulsations are detected) but no bursts are observed, we can constrain the distance by adopting a composition for the fuel and comparing the predictions of burst models to the good-time intervals of our X-ray data.

For example, based on {\it RXTE}/PCA observations of the accretion-powered millisecond pulsar IGR~J00291+5934, \cite{gal06b} derived joint constraints over the distance $d$ and fuel H-fraction $X_0$, adopting  the predictions of a simple numerical model ({\it ref to {\sc settle}}). These constraints suggest the distance is $\gtrsim5$~kpc (at 3-sigma significance), provided $X_0\approx0.7$, a reasonable choice given the expectation of a H-rich donor in this $2.46$~hr binary orbital period system \cite[]{gal05a}.  \todo{give more detail about how this was done}

In a subsequent outburst in 2015, a single burst was detected by {\it Swift}/XRT \cite[]{kuin15}, offering the opportunity to verify the previously-determined distance limit. The estimated peak 
% 0.3--10~keV flux in the range (3.0--$5.6)\times10^{-11}\ \epcs$. 
0.1--35~keV flux was $(18\pm4)\times10^{-8}\ \epcs$ \cite[]{defalco17}, with the burst exhibiting spectral variations indicative of PRE. By comparing the peak flux to the empirical Eddington luminosity (see \S\ref{subsec:1burst}, below), the corresponding distance 
\cite[including the inclination range of 22--32$^\circ$ suggested by][]{torres08},
% possible inclinations as described in sec \S\ref{subsec:data}) is $4.4_{-0.7}^{+0.8}$~kpc, which is mostly inconsistent with the earlier limit. Clearly, such limits must be viewed with some caution.
is $5.0_{-0.5}^{+0.7}$~kpc, which is fully consistent with the previously established limit. The effect of the inclination constraints is illustrated in Fig. \ref{fig:dist_igr00291}).

\begin{figure}[ht!]
\plotone{fig1.pdf}
\caption{Estimated distance distributions for the accretion-powered X-ray pulsar IGR~J00291+5934, based on the peak flux of the sole thermonuclear burst observed by \cite{defalco17}.
%
The distribution for an isotropic distribution of system inclinations (up to a maximum value of $75^\circ$, motivated by the lack of dips in the X-ray intensity) is shown as the blue shaded histogram. Imposing the inclination constraint of 22--32$^\circ$ suggested by \cite{torres08}, for which the predicted burst anisotropy factor is $\xi_b=0.704$ on average, gives instead the orange-shaded histogram. The resulting $1\sigma$ confidence interval is $5.0_{-0.5}^{+0.7}$~kpc.
\label{fig:dist_igr00291}}
\end{figure}

\subsection{One burst -- estimating the burst rate} 
\label{subsec:1burst}

% {\it copy the analysis for the IGR source here }

For a single burst detected from a source, we can expect to measure the burst peak flux, fluence, and timescale, and infer the distance and hence the expected burst rate. The detailed shape of the lightcurve can provide constraints from comparisons with numerical models (see \S\ref{subsec:lccompare}), but here we focus on the more straightforward calculation which may be done from the simple burst measurements alone. 

We use for example the sole burst detected with \igr/JEM-X from IGR~J17591$-$2342 \cite[]{kuiper20}. This burst was observed  when the estimated persistent bolometric flux was $(1.2\pm0.2)\times10^{-9}\ \epcs$, and exhibited a fluence of $E_b =(1.1\pm0.1)\times10^{-6}\ {\rm erg\,cm^{-2}}$. The peak flux was $(7.6\pm1.4)\times10^{-6}\ \epcs$.

\cite{kuiper20} first estimated the distance to the source based on the empirical value of the Eddington luminosity, and incorporating the expected anisotropy for an inclination range of 
% kuiper20 uses slightly more restrictive 28-30
24--30$^\circ$ \cite[]{sanna18} as $7.7_{-0.6}^{+0.8}$~kpc. Here the uncertainty is dominated by the peak flux, since the anisotropy factor is effectively fixed for such a narrow range of inclination, at $\xi_b=0.70$.

\todo{potentially also demonstrate here the effect of inclination}

We can use this distance constraint to estimate in turn the accretion and burst rate, as follows.
%
The persistent flux can then be converted to an accretion rate, given the estimated anisotropy factor predicted (separately) for the persistent emission, as
\begin{eqnarray}
\dot{m} & = L_{\rm pers} (1+z) (4\pi R^2(GM/R_{\rm NS}))^{-1}\\
& = \ldots
% TODO: give expanded expression for this and move to earlier section
\end{eqnarray}
via the {\tt mdot} method, and incorporating the estimated distance distribution already determined. The estimated value is within the range $(2.0_{-0.4}^{+0.6})\times10^3\ \rm{g\,cm^{-2}\,s^{-1}}$ with the usual assumed range for neutron star mass and radius (and hence redshift).

In parallel, we can estimate the ignition column using equation \ref{eq:column}, given an estimate of the nuclear burning yield, $\qnuc$. ; as before, we could potentially obtain an estimate of this quantity by comparison of the observed lightcurve to simulations. However, in the absence of an estimate of this quantity, all is not lost. 

Instead we can work backwards, 
% 
and assume a set of randomly generated values of \Xb\
to estimate $\yign$ from equation \ref{column}, and $\Delta t_{\rm rec} = \yign/\dot{m}$.
%
We choose a uniform distribution for \Xb\ in the range 0--0.7; this distribution is probably not realistic for a wide sample of bursts, but it is difficult to do better.
% text copied from Kuiper et al. 2020 down to ********
% not yet edited (need to modify sufficiently to avoid plagiarism issues
Given each pair of (\Xb,  $\Delta t_{\rm rec}$) we can infer the accreted H-fraction $X_0$, based on 
% the expectation that hydrogen burns via the hot-CNO cycle and will be exhausted in a time $t_{\rm CNO} =9.8(X_0/0.7)(Z_{\rm CNO}/0.02)^{-1}$ \citep[e.g.][]{lampe16}.
equation \ref{eq:X_0}.
%
Since $Z_{\rm CNO}$ is also unknown, we similarly draw random values from a uniform distribution in the range 0.0--0.02.
We then select only those $X_0$ values within a physically realistic range $X_0<0.75$, and compute the confidence limits on the remaining parameters. 

For the single burst observed from IGR~J17591$-$2342 the inferred ignition column is $2.2_{-0.8}^{+1.4}\times10^8\,{\rm g\,cm^{-2}}$, and the average expected recurrence time is $1.8\pm0.7$~d. 
% Such long recurrence times likely guarantee that the hydrogen in the accreted fuel is exhausted at the base prior to ignition, and explaining the fast rise and short duration of the burst. 
We can also infer lower (upper) limits on $X_0$ ($Z_{\rm CNO}$), although these limits are not strongly constraining; we find $X_0>0.17$ and $Z_{\rm  CNO}<0.017$ at 95\% confidence.
% *******

% Figure produced from the notebook; used 100000 samples for a smoother set of 
% distributions
\begin{figure}[ht!]
\plotone{igrJ17591-2342.pdf}
\caption{Inferred distributions of burst parameters for the single event observed from IGR~J17591$-$2342 with \igr/JEM-X, as reported by \cite{kuiper20}. An initially uniform distribution of \Xb, the mean H fraction at ignition, is used to estimate the burst ignition column $\yign$\ from the measured fluence, $E_b$. The column $\yign$\ and accretion rate are then used to estimate the recurrence time $\Delta t$ and the accreted H-fraction $X_0$, and we select only the physically realistic values $X_0<0.75$. The resulting constraints on the fuel composition parameters $X_0$ and $\zcno$ are rather weak, but the recurrence time is somewhat better constrained.
\label{fig:param_igr17591}}
\end{figure}

% ------------------------------------------------------------------------------
\subsection{Two bursts -- inferring fuel composition} 
\label{subsec:fuelcomp}

Given a minimally complete set of burst observations as described in \S\ref{subsec:data}, a common approach \cite[e.g][]{falanga11} is to estimate the fuel composition at ignition, and hence the accreted fuel composition, based on simple analytic estimates of the burst energy production $Q_{\rm nuc}$.

One issue with this approach (as for analyses based on one burst, as described in \S\ref{subsec:1burst}) is that the effects of anisotropy are frequently neglected.
%
A second issue which has emerged recently is that the commonly used estimates for the nuclear energy generation rate $Q_{\rm nuc}$ are not supported by recent {\sc kepler} models. 
%
We describe the general approach below including the effects of these two issues.

Given the measured $\alpha$-value, the nuclear energy generation rate $Q_{\rm nuc}$ can be estimated, as follows:
\begin{eqnarray}
\alpha & = & \frac{\Delta t F_{\rm per} c_{\rm bol}}{E_B} \\
& = & \frac{Q_{\rm grav}}{Q_{\rm nuc}}\frac{\xi_b}{\xi_p}(1+z)
\end{eqnarray}
where $Q_{\rm grav} = c^2z/(1+z) \approx GM_{\rm NS}/R_{\rm NS}$ is the gravitational energy release per gram. Note the dependence of the observational $\alpha$-value on both the surface redshift and the anisotropy parameters. The nuclear energy generation rate is thus
\begin{equation}
Q_{\rm nuc} = \frac{c^2 z}{\alpha}\frac{\xi_b}{\xi_p} \label{qnuc}
\end{equation}
Thus, given an estimate of $z$ and the inclination, the burst measurements can be used to infer $Q_{\rm nuc}$. 

% describe plausible assumptions for ration of $\xi_b/\xi_p$.
The system inclination for LMXBs is notoriously difficult to measure. For non-dipping sources, the likely range is up to $72^\circ$, while dipping sources are likely $i\gtrsim79.3^\circ$ \cite[]{gal16a}.
% 
% 5, 16, 50, 68, 95 th percentiles:
% [ 0.58815149  0.72522467  1.12987208  1.49149624  1.59287602]
% min, max
% 0.525409532417 1.63715782279
For an isotropically-distributed population of non-dipping sources, the ratio $\xi_b/\xi_p$ is roughly uniformly-distributed between 0.53--1.64, rising a little towards the high end. The median value is 1.13, while the 95\% confidence interval is 0.73--1.49.

\begin{itemize}
\item describe plausible assumptions for $z$, e.g. based on recent analysis of bursts
\end{itemize}

Given an estimate of $Q_{\rm nuc}$, the average hydrogen fraction $\langle X\rangle$ in the fuel layer at ignition has been estimated to date using the relation $Q_{\rm nuc}=1.6 + 4\langle X\rangle$~MeV/nucleon \cite[e.g.][and references therein]{gal03d}. This expression includes $\approx35$\% losses
attributed to neutrino emission \cite[]{fuji87}. However, the 35\% value applies only to $\beta$-decays in the much more extensive {\it rp}-process burning chain, and recent {\sc kepler} simulations indicate instead an approximation of 
% $Q{\rm nuc}=0.96+6.6\langle X\rangle$ 
$Q{\rm nuc}=1.35+6.05\langle X\rangle$ 
% (Goodwin et al. 2018, in preparation). 
\cite[]{goodwin19a}.
[{\it comment on improved quadratic expression}]
%
Thus, by substituting in equation \ref{qnuc} we can estimate 
\begin{equation}
\langle X\rangle 
% In [42]: c.c**2/(6.0455*u.MeV/c.m_p).to("m**2/s**2")                             
% Out[41]: <Quantity 155.20173377>
%% = \frac{c^2z/\alpha}{6.6\ {\rm Mev/nucleon}}\frac{\xi_b}{\xi_p} - 0.145
% = z\frac{142}{\alpha}\frac{\xi_b}{\xi_p} - 0.145 \label{xbar}
 = z\frac{155}{\alpha}\frac{\xi_b}{\xi_p} - 0.223 \label{xbar}
\end{equation}
[{\it comment on the degree of error introduced by ignoring anisotropy etc.}] 
%
Clearly this expression is only applicable for $\alpha$-values up to some limit, which we calculate as
\begin{equation}
%In [43]: c.c**2/(1.*u.MeV/c.m_p).to("m**2/s**2")/1.3455        
% Out[43]: <Quantity 697.34082607>
% \alpha \leq 980\, z \frac{\xi_b}{\xi_p}
\alpha \leq 697\, z \frac{\xi_b}{\xi_p}
\end{equation}
Interestingly, this limit is 
% $\approx70$\% 
%  c.c**2/(1.*u.MeV/c.m_p).to("m**2/s**2")/1.6                 
% Out[44]: <Quantity 586.42005092>
% In [45]: 697./586.                                                               
% Out[45]: 1.189
$\approx20$\%
%
larger than the equivalent value derived for the old expression for $Q_{\rm nuc}$. This result suggests that larger observed values of $\alpha$ may be accommodated without resorting to explanations including incomplete burning of burst fuel \cite[e.g.][]{bcatalog}.

% upper limit on xbar
Conversely, for low values of $\alpha$, we may apply the constraint that $\langle X\rangle \lesssim 0.77$, corresponding to the expected maximum possible for accreted fuel with primordial abundances.

The hydrogen fraction at ignition $\langle X\rangle$ and the burst recurrence time may then be used to estimate the hydrogen fraction in the fuel, $X_0$. The H-fraction at the base of the fuel layer is reduced steadily by $\beta$-limited hot CNO burning, and will completely exhaust the accreted hydrogen in a time \cite[]{lampe16}:
\begin{equation}
t_{\rm CNO} =9.8\left(\frac{X_0}{0.7}\right) \left(\frac{Z_{\rm CNO}}{0.02}\right)^{-1}
\end{equation}
measured in the neutron star frame.
% updated 2019 Aug 26 with exhaustion case
The hydrogen fraction at the base will thus be $X_0[1-\Delta t/(1+z)t_{\rm CNO}]$, 
% [{\it should this expression include time dilation?}] -- yes
%
provided that $\Delta t < (1+z)t_{\rm CNO}$. Once $\Delta t$ exceeds the time to burn all the hydrogen at the base, the abundance there will be zero, and a growing layer of pure He fuel will develop. The average H-fraction in the layer for these two cases will be 
%so that the average H-fraction in the layer will be 
\begin{equation}
\langle X \rangle = \begin{cases}
%  X_0\left[1-0.5\frac{\Delta t}{(1+z)t_{\rm CNO}}\right] & \Delta t \leq (1+z)t_{\rm CNO} \\
%  0.5X_0\frac{(1+z)t_{\rm CNO}}{\Delta t} & \Delta t > (1+z)t_{\rm CNO}
  X_0(1-0.5f_{\rm burn}) & \Delta t \leq (1+z)t_{\rm CNO} \\
  0.5X_0/f_{\rm burn} & \Delta t > (1+z)t_{\rm CNO}
  \end{cases}
\end{equation}
where $f_{\rm burn} = \frac{\Delta t}{(1+z)t_{\rm CNO}}$ is the ratio of the recurrence time to the time to burn all the H.
%
We can combine this expression with equation \ref{xbar} to give
\begin{equation}
X_0 = \begin{cases}
%        & \langle X\rangle +0.35\left[\frac{\Delta t}{(1+z)9.8\ {\rm hr}}\right]
%                            \left(\frac{Z_{\rm CNO}}{0.02}\right) \nonumber \\
         z\frac{142}{\alpha}\frac{\xi_b}{\xi_p} - 0.145 
                           +\left[\frac{\Delta t}{(1+z)28\,{\rm hr}}\right]
                       \left(\frac{Z_{\rm CNO}}{0.02}\right) & f_{\rm burn} \leq 1\\
         \sqrt{ \frac{\Delta t}{(1+z)7\,{\rm hr}}\frac{Z_{\rm CNO}}{0.02}
                        \left(z\frac{142}{\alpha}\frac{\xi_b}{\xi_p} - 0.145\right) }
                        & f_{\rm burn} > 1
        \end{cases}
    \label{eq:X_0}
\end{equation}
Practically, the issue with these expressions is that calculating $f_{\rm burn}$ requires knowledge of $X_0$, which is the unknown we are trying to constrain. One approach is to adopt a trial value of $X_0$, calculate $f_{\rm burn}$ and hence an updated estimate of $X_0$ via equation \ref{x0}, and iterate until no further change in the estimate arises.

Clearly this estimate of $X_0$ will incorporate systematic errors arising from the assumed values of $z$, the inclination, and $Z_{\rm CNO}$. We explore the typical range of these errors in 
% the next section.
\S\ref{sec:simulations}

\begin{itemize}
\item Apply the method to the reference bursts
\end{itemize}

% comment on the inability to measure distance/mass/radius? Or can we actually do that?
% I think you can get distance by comparing the ignition column with the fluence

\subsection{Inferring distance} 
\label{subsec:distance}

The observed burst energy permits estimation of the ignition depth, assuming complete fuel consumption. For a burst fluence $E_b$, the total burst energy release (taking into account the anisotropy effect) is
\begin{equation}
E_{\rm burst} = 4\pi d^2 \xi_b E_b 
\end{equation}
The ignition depth is given by 
\begin{eqnarray}
y_{\rm ign} & = & E_{\rm burst}\frac{1 + z}{4\pi R_{\rm NS}^2 Q_{\rm nuc}} \\
  & = & E_b \frac{\xi_b (1 + z)}{ Q_{\rm nuc}} \left(\frac{d}{R_{\rm NS}}\right)^2 
\end{eqnarray}
Independently, we can estimate the accreted mass as simply $\Delta M = ...$ and thus estimate the distance...?

\subsection{Observation-model comparisons} 
\label{subsec:lccompare}

The model-predicted quantities must then be converted to what a distant observer would see, by taking into account the effects of source distance, emission anisotropy, general relativistic time dilation and redshift. For the purposes of comparison, we convert the model lightcurves to what a distant observer would see, following the approach described in \S\ref{subsec:simobs}.


For a given pair of observed and model lightcurves, the only parameters that affect the comparison are the source distance $d$, the anistotropy parameters $\xi_b$, $\xi_p$ (each a function of the inclination $i$), and the gravitational redshift $1+z$ (which also determines the parameter $\xi$).
%
We also introduce a ``nuisance'' parameter, $t_{\rm off}$, which is required to align the observed and predicted model lightcurve so as to minimise any residual differences. This parameter and the (time dilated) model timestamps $t_{i,\infty}$ are used to overlay the model predicted lightcurve onto the observed one.

Our approach is then to explore the parameter space of $(d, i, 1+z, t_{\rm off})$ to find the best set of parameters for which the comparison likelihood is maximised:
\begin{eqnarray}
\mathcal{L} & = & -f_F\left[\left( \frac{F_p-F_{p,\rm{inf}}}{\sigma_{p}}\right)^2 
    + \log\left(\frac{2\pi}{\sigma_p}\right)\right] \\
 & & -f_t\left[ \left( \frac{\Delta t-\Delta t_{{\rm pred},\infty}}{\sigma_t}\right)^2
    + \log\left(\frac{2\pi}{\sigma_t}\right)\right] + ...\\
\end{eqnarray}
where $f_F$...

Varying the redshift $1+z$ will allow us to obtain the best match between the model and observed lightcurve.
%this parameter cannot be varied arbitrarily, 
As each model run has been performed with a particular value of the surface gravity $g$, a particular value of $1+z$ implies in turn specific values of $M_{\rm NS}$ and $R_{\rm NS}$. 
%We take the approach of keeping $R_{\rm NS}$ fixed and allowing $M_{\rm NS}$ to vary to give the required change in $1+z$, since many equations of state have roughly constant radii over a range of masses \cite[e.g.][]{lp07}.

\begin{itemize}
\item Describe Zac's method
\item Describe my method (via multinest)
\end{itemize}

%\startlongtable
%\begin{deluxetable}{c|cc}
%\tablecaption{ApJ costs from 1991 to 2013\tablenotemark{a} \label{tab:table}}
%\tablehead{
%\colhead{Year} & \colhead{Subscription} & \colhead{Publication} \\
%\colhead{} & \colhead{cost} & \colhead{charges\tablenotemark{b}}\\
%\colhead{} & \colhead{(\$)} & \colhead{(\$/page)}
%}
%\colnumbers
%\startdata
%1991 & 600 & 100 \\
%1992 & 650 & 105 \\
%1993 & 550 & 103 \\
%1994 & 450 & 110 \\
%1995 & 410 & 112 \\
%1996 & 400 & 114 \\
%1997 & 525 & 115 \\
%1998 & 590 & 116 \\
%1999 & 575 & 115 \\
%2000 & 450 & 103 \\
%2001 & 490 &  90 \\
%2002 & 500 &  88 \\
%2003 & 450 &  90 \\
%2004 & 460 &  88 \\
%2005 & 440 &  79 \\
%2006 & 350 &  77 \\
%2007 & 325 &  70 \\
%2008 & 320 &  65 \\
%2009 & 190 &  68 \\
%2010 & 280 &  70 \\
%2011 & 275 &  68 \\
%2012 & 150 &  56 \\
%2013 & 140 &  55 \\
%\enddata
%\tablenotetext{a}{Adjusted for inflation}
%\tablenotetext{b}{Accounts for the change from page charges to digital quanta in April, 2011}
%\tablecomments{Note that {\tt \string \colnumbers} does not work with the 
%vertical line alignment token. If you want vertical lines in the headers you
%can not use this command at this time.}
%\end{deluxetable}


%\begin{deluxetable*}{ccCrlc}[b!]
%\tablecaption{Column math mode in an observation log \label{tab:mathmode}}
%\tablecolumns{6}
%\tablenum{2}
%\tablewidth{0pt}
%\tablehead{
%\colhead{UT start time\tablenotemark{a}} &
%\colhead{MJD start time\tablenotemark{a}} &
%\colhead{Seeing} & \colhead{Filter} & \colhead{Inst.} \\
%\colhead{(YYYY-mm-dd)} & \colhead{(d)} &
%\colhead{(arcsec)} & \colhead{} & \colhead{}
%}
%\startdata
%2012-03-26 & 56012.997 & \sim 0.\arcsec5 & H$\alpha$ & NOT \\
%2012-03-27 & 56013.944 & 1.\arcsec5 & grism & SMARTS \\
%2012-03-28 & 56014.984 & \nodata & F814M & HST \\
%2012-03-30 & 56016.978 & 1.\arcsec5\pm0.25 & B\&C & Bok \\
%\enddata
%\tablenotetext{a}{At exposure start.}
%\tablecomments{The ``C'' command column identifier in the 3 column turns on
%math mode for that specific column. One could do the same for the next
%column so that dollar signs would not be needed for H$\alpha$
%but then all the other text would also be in math mode and thus typeset
%in Latin Modern math and you will need to put it back to Roman by hand.
%Note that if you do change this column to math mode the dollar signs already
%present will not cause a problem. Table \ref{tab:mathmode} is published 
%in its entirety in the machine readable format.  A portion is
%shown here for guidance regarding its form and content.}
%\end{deluxetable*}


%%% Note that the \setcounter and \renewcommand are needed here because
%%% this example is using a mix of deluxetable and tabular.  Here the
%%% deluxetable counters are set with \tablenum but the situation is a bit
%%% more complex for tabular.  Use the first command to set the Table number
%%% to ONE LESS than it should be.  The next command will auto increment it
%%% to the desired number.
%\setcounter{table}{2}
%\begin{table}[h!]
%\renewcommand{\thetable}{\arabic{table}}
%\centering
%\caption{Decimal alignment made easy} \label{tab:decimal}
%\begin{tabular}{cD@{$\pm$}D}
%\tablewidth{0pt}
%\hline
%\hline
%Column & \multicolumn2c{Value} & \multicolumn2c{Uncertainty}\\
%\hline
%\decimals
%A & 1234     & 100.0     \\
%B &  123.4   &  10.1     \\
%C &  12.34   &   1.01    \\
%D &   1.234  &   0.101   \\
%E &    .1234 &   0.01001 \\
%F &   1.0    &    .      \\
%\hline
%\multicolumn{5}{c}{NOTE. - Two decimal aligned columns}
%\end{tabular}
%\end{table}

%%% The "ht!" tells LaTeX to put the figure "here" first, at the "top" next
%%% and to override the normal way of calculating a float position
%\begin{figure}[ht!]
%\plotone{cost.eps}
%\caption{The subscription and author publication costs from 1991 to 2013.
%The data comes from Table \ref{tab:table}.\label{fig:general}}
%\end{figure}

%\begin{equation}
%\bar v(p_2,\sigma_2)P_{-\tau}\hat a_1\hat a_2\cdots
%\hat a_nu(p_1,\sigma_1) ,
%\end{equation}

%\begin{eqnarray}
%\gamma^\mu  & = &
% \left(
%\begin{array}{cc}
%0 & \sigma^\mu_+ \\
%\sigma^\mu_- & 0
%\end{array}     \right) ,
% \gamma^5= \left(
%\begin{array}{cc}
%-1 &   0\\
%0 &   1
%\end{array}     \right)  , \\
%\sigma^\mu_{\pm}  & = &   ({\bf 1} ,\pm \sigma) , 
%\end{eqnarray}

%\begin{eqnarray}
%\hat a & = & \left(
%\begin{array}{cc}
%0 & (\hat a)_+\\
%(\hat a)_- & 0
%\end{array}\right), \nonumber \\
%(\hat a)_\pm & = & a_\mu\sigma^\mu_\pm 
%\end{eqnarray}

\section{Application to simulated data} 
\label{sec:simulations}

Here we assemble and analyse a number of datasets to assess the precision and accuracy of the approaches described in the previous section.

%% If you wish to include an acknowledgments section in your paper,
%% separate it off from the body of the text using the \acknowledgments
%% command.
\acknowledgments

To be added

%% To help institutions obtain information on the effectiveness of their 
%% telescopes the AAS Journals has created a group of keywords for telescope 
%% facilities.
%
%% Following the acknowledgments section, use the following syntax and the
%% \facility{} or \facilities{} macros to list the keywords of facilities used 
%% in the research for the paper.  Each keyword is check against the master 
%% list during copy editing.  Individual instruments can be provided in 
%% parentheses, after the keyword, but they are not verified.

\vspace{5mm}
\facilities{HST(STIS), Swift(XRT and UVOT), AAVSO, CTIO:1.3m,
CTIO:1.5m,CXO}

%% Similar to \facility{}, there is the optional \software command to allow 
%% authors a place to specify which programs were used during the creation of 
%% the manusscript. Authors should list each code and include either a
%% citation or url to the code inside ()s when available.

\software{astropy \citep{astropy13}  
%          Cloudy \citep{2013RMxAA..49..137F}, 
%          SExtractor \citep{1996A&AS..117..393B}
          }

%% Appendix material should be preceded with a single \appendix command.
%% There should be a \section command for each appendix. Mark appendix
%% subsections with the same markup you use in the main body of the paper.

%% Each Appendix (indicated with \section) will be lettered A, B, C, etc.
%% The equation counter will reset when it encounters the \appendix
%% command and will number appendix equations (A1), (A2), etc. The
%% Figure and Table counter will not reset.

%\appendix
%
%\section{Appendix information}

%% The reference list follows the main body and any appendices.
%% Use LaTeX's thebibliography environment to mark up your reference list.
%% Note \begin{thebibliography} is followed by an empty set of
%% curly braces.  If you forget this, LaTeX will generate the error
%% "Perhaps a missing \item?".
%%
%% thebibliography produces citations in the text using \bibitem-\cite
%% cross-referencing. Each reference is preceded by a
%% \bibitem command that defines in curly braces the KEY that corresponds
%% to the KEY in the \cite commands (see the first section above).
%% Make sure that you provide a unique KEY for every \bibitem or else the
%% paper will not LaTeX. The square brackets should contain
%% the citation text that LaTeX will insert in
%% place of the \cite commands.

%% We have used macros to produce journal name abbreviations.
%% \aastex provides a number of these for the more frequently-cited journals.
%% See the Author Guide for a list of them.

%% Note that the style of the \bibitem labels (in []) is slightly
%% different from previous examples.  The natbib system solves a host
%% of citation expression problems, but it is necessary to clearly
%% delimit the year from the author name used in the citation.
%% See the natbib documentation for more details and options.

\bibliography{all}
\bibliographystyle{apj}

%% This command is needed to show the entire author+affilation list when
%% the collaboration and author truncation commands are used.  It has to
%% go at the end of the manuscript.
%\allauthors

%% Include this line if you are using the \added, \replaced, \deleted
%% commands to see a summary list of all changes at the end of the article.
%\listofchanges

\end{document}

% End of file `sample61.tex'.
